\label{subsec:estione}

Besides the age at school closure, the impact of school closure on educational attainment may also differ by the number of years of exposure to the policy: short-run effects of closure on a child's educational attainment progression could be dampened or amplified over the medium and long run.\footnote{After an individual completes schooling, duration effects will become constant. In studies with cross-sectional data taken long after a policy has been implemented, \textcite{duflo_2001_school} for example, the duration effect is irrelevant because all educational attainment data is observed long after sample individuals have completed schooling. In our data, a significant proportion of individuals have not completed schooling, allowing us to have meaningful duration effects.}  In order to identify both age and duration effects with our cross-sectional data, we exploit the variation in the year of school closure. Under the assumption that the impact of the policy is not specific to the calendar year of closure, we can estimate Equation \ref{eq:startsLengths} to obtain the impact of the policy as a function of both starting age and the length of exposure.

In Equation \ref{eq:startsLengths}, we use similar notations as in Equation \ref{eq:startsOnly}, the difference is that the policy's effects are now captured by $\hat{\lambda_{zl}}$ that varies by age-at-closure variable \(t_{i}\) and years-of-exposure variable $\tau_{i}$:

\begin{singlespace}\vspace*{-\baselineskip}
\begin{eqnarray}
E_{pvia} & = & \phi+\beta_{v}+\rho_{pa} \nonumber \\
&  & + \hat{\lambda_{zl}} \cdot \mathbb{1}\left\{ (l_{l}\leq\tau_{i}\leq u_{l} ) \cap  ( l_{z}\leq t_{i}\leq u_{z} ) \right\} \cdot c_{v} \label{eq:startsLengths}\\
&  & +X_i\cdot\gamma\nonumber +\epsilon_{i}\nonumber
\end{eqnarray}
\end{singlespace}\noindent\ignorespaces
where, as before, \(c_{v}\) is a binary variable indicating if individual \(i\) is from a village \(v\) with school consolidation (i.e. treatment village). As in Equation \ref{eq:startsOnly}, we group children in villages with school closure into $Z$ groups based on their age at closure, with lower and upper bounds for each group, $l_{z}$ and $u_{z}$. To capture duration effects, we further divide each of the Z groups of children into $L$ groups based on the length of exposure $\tau_i$, defined as the gap between individual $i$'s age in 2011 and $i$'s age at year of school closure, $t_{i}$.\footnote{$\tau_i = \min(a_i, a_i - t_i)$: \(\tau_i\) is the gap between age in 2011 and $t_{i}$ if individual $i$ was borne before the year of closure, and it is the age of the child in 2011 if the child was borne after school closure. } Each $l$ length of exposure group includes those with $\tau_i$ falling within lower and upper bounds, $l_{l}$ and $l_{u}$. The exposure groups allow us to separately estimate the short, medium and long run effects of the consolidation policy on educational attainment. There are $Z\cdot L$ groups of interest for this regression.\footnote{Ideally, we would estimate the policy effects for each $t_i$ and $\tau_i$ combination separately, but we have constructed the $Z$ and $L$ groups due to limited sample size.}
