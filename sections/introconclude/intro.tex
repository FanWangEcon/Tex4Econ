\newcommand{\introfootone}{For example, \textcite{gine_access_2011, karaivanov_disadvantages_2018} use the initial waves of the Townsend Thai village survey to study how household firms choose between formal and informal borrowing in two-period models. \textcite{gine_access_2011} distinguishes formal and informal borrowing by interest rates, fixed costs, and collateral constraints.}

It is well known that in village economies, local informal financial arrangements exist in the absence of external formal credit market options \autocite{udry_risk_1994, townsend_risk_1994}. As formal borrowing and savings opportunities have expanded, informal credit markets have persisted. But dynamic models that study the effects of financial deepening on development do not explicitly consider informal financial options, and studies that test the fit of informal risk-sharing models to data generally do not consider formal options explicitly.\footnote{See \textcite{townsend_financial_2010, buera_entrepreneurship_2015} for reviews.} On the other hand, studies which explicitly analyze the interaction of formal and informal credit markets generally do so in non-dynamic settings.\footnote{\introfootone} It is difficult to fully analyze the effects of financial deepening which determines asset distributions, however, when these distributions are fixed. In this paper, I develop and estimate a dynamic equilibrium risky entrepreneur model that incorporates formal as well as informal borrowing and savings choices. In this model, households are infinitely-lived, risk-averse, and have varying productivity and \CZH.
