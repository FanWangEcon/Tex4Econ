The analysis in this paper contributes to several strands of the literature. First, there has been substantial research on the impacts of greater financial access on developing economies \autocite{greenwood_financial_1990, banerjee_occupational_1993, lloyd-ellis_enterprise_2000, gine_evaluation_2004, greenwood_financing_2010, kaboski_structural_2011, moll_productivity_2014, buera_macroeconomics_2012, dabla-norris_distinguishing_2018}. Despite the importance of informal financial arrangements \autocite{udry_risk_1994, townsend_risk_1994}, these dynamic models of financial deepening--formal credit market expansion--generally do not explicitly consider informal financial options.\footnote{\textcite{banerjee_credit_2017} allow households to borrow from formal and informal sectors at exogenous rates to finance within period capital investments, and households can save across periods at an exogenous negative savings rate.} Additionally, studies that test the fit of informal risk-sharing models to data do not model formal options explicitly \autocite{alem_evaluation_2014, karaivanov_dynamic_2014, kinnan_distinguishing_2017}. In this paper, I model risky entrepreneurs' choices over formal and informal credit market options in an exogenous incomplete markets setting.\footnote{While there are different ways for rural households to transfer financial resources, \textcite{karaivanov_dynamic_2014} find that a model with exogenously incomplete borrowing and savings options fit consumption and investment data in rural Thai villages better than constrained efficient credit/insurance models. The model in effect augments equilibrium models of risky entrepreneurs (see review: \textcite{quadrini_entrepreneurship_2009}) with additional exogenous borrowing and savings options.} Similar to \textcite{kaboski_structural_2011}, I treat villages as small open economies where formal prices are exogenously determined, but I extend the framework to explicitly consider informal choices and equilibrium interest rates determined within each local informal credit market. My approach here focuses on the \textit{micro-equilibrium} effects of formal credit market expansion on village credit markets. This is different from \textcite{buera_macroeconomics_2012, breza_measuring_2018}, which study the macro equilibrium effects of large microfinance roll-outs on prices, including interest rates and wages, in the aggregate economy.

Second, this paper contributes to works that study the interaction between formal and informal credit markets. Third, there is a significant and growing empirical literature that analyzes separate dimensions of credit market policies. Fourth, there is a literature that studies how the provision of formal insurance could crowd-out informal insurance \autocite{attanasio_consumption_2000,krueger_public_2011,chandrasekhar_information_2011}.

The structure of the paper is as follows. Section \ref{sec:modelmodel} develops the model. Section \ref{sec:simusimu} describes model mechanisms and demonstrates the equilibrium effects of shifting various  dimensions of formal and informal credit market access costs. In Section \ref{sec:Data-and-Background}, I describe the data and background. Section \ref{sec:estiesti} describes estimation results and counterfactuals. I offer the conclusion in Section \ref{sec:concludeconclude}. Additional details for the solution and estimation methods are in Sections \ref{sec:solualgo}.
