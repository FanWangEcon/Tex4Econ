
In recent decades, \NLC financial services have expanded significantly in developing countries. This paper evaluates the impacts of improving access to the \NLC credit market on rural households, taking into consideration the impacts of changing \NLC credit market conditions on the \LLC credit market.

I built a risky entrepreneur model assuming that villages are small open economies with respect to \NLC credit market options, but households can also borrow and save in an equilibrium local credit market. I showed that \NLC credit market expansions through interest rates subsidies, access fixed costs reductions, and collateral constraint relaxations have heterogeneous and non-separable effects on households. These effects differ depending on informal credit market conditions. In the Thai case, villages already had extensive \LLC \BBBB and \SSSS activities, and the effects of \NLC credit market expansions on household welfare were hence limited.
