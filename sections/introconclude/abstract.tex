% JEL CODES
\newcommand{\jelcodes}{D15, D25, G21, O16, O17}
\newcommand{\keywords}{Informal credit, microfinance, structural estimation}
% Informal credit, Family loans, Social capital, Limited enforcement, Default risk
% Microfinance, Entrepreneurship, Social Networks
% Microfinance, entrepreneurship, general equilibrium effectiveness
%% 2018-12-13 05:38
\newcommand{\abstractA}{I develop a dynamic equilibrium model in which risky entrepreneurs sort among local-informal and outside-formal borrowing and savings opportunities. A local interest rate clears the informal credit market, but villages are small-open economies with respect to the formal market. Identical policies that shift the interest rate, fixed cost or collateral constraint of formal access have non-linear and non-separable effects depending on the costs of informal access. I develop computational methods for solving and estimating dynamic equilibrium models with discrete and continuous choices that leverage scalable parallel cloud computing resources. I take the model to villages in Thailand which implemented formal credit access expansion policies, and estimate changes in the costs and constraints of access from changes in credit market participation. I find that lower fixed costs increased the proportion of households borrowing formally, and that relaxed collateral constraints lowered informal interest rates. In terms of welfare, I find low wealth but productive households benefited from policies that expanded credit access, but average gains are smaller than would be predicted by models that do not consider the substantial informal credit market. Moreover, approximately 18\% of households suffered welfare losses because of diminished opportunities for informal saving.}

\newcommand{\abstractB}{I develop and estimate a dynamic equilibrium model of risky entrepreneurs' borrowing and savings decisions incorporating both formal and local-informal credit markets. Households have access to an exogenous formal credit market and to an informal credit market in which the interest rate is endogenously determined by the local demand and supply of credit. I estimate the model via Simulated Maximum Likelihood using Thai village data during an episode of formal credit market expansion. My estimates suggest that a 49 percent reduction in fixed costs increased the proportion of households borrowing formally by \FIPNafterFCPropIncRnd\xspace percent, and that a doubling of the collateralized borrowing limits lowered informal interest rates by \IRNafterFCPropIncRnd\xspace percent. I find that more productive households benefited from the policies that expanded borrowing access, but less productive households lost in terms of welfare due to diminished savings opportunities. Gains are overall smaller than would be predicted by models that do not consider the informal credit market. \textbf{JEL codes}: \jelcodes}
