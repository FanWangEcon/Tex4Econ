\documentclass[12pt,english]{article}

% Basic packages
\usepackage{graphicx}
\usepackage{lipsum}
\usepackage{blindtext}
\usepackage{listings}

% Margins 
\usepackage{geometry}
\geometry{
	a3paper,
	noheadfoot=true,
	left=1.0in,
	right=1.0in,
	top=1.0in,
	bottom=1.0in,
}

% url package
\usepackage[colorlinks=true,
            linkcolor=blue,
            urlcolor=blue,
            anchorcolor = blue,
            citecolor=gray]{hyperref}

% Specific Packages
\usepackage{float}
\usepackage{subcaption}
\usepackage{caption}
\captionsetup[figure]{labelfont={bf},name={Fig.},labelsep=period}

% Titling and Author
\title{Latex Examples, Figure Vertical and Horizontal Alignments with Subfigures}
\author{\href{https://fanwangecon.github.io/}{Fan Wang}\thanks{https://fanwangecon.github.io, repository: \href{https://fanwangecon.github.io/Tex4Econ/}{Tex4Econ}}}
\date{\today}
\begin{document}

\maketitle
\tableofcontents
\clearpage 

\section{Vertical center align with subfigures}

The figure has the following features:
\begin{itemize}
    \item An overall caption with width control
    \item Below figure notes with width control
    \item subfigure vertically center aligned
\end{itemize}

\lstset{%
  basicstyle=\footnotesize\ttfamily,
  breaklines=true,
  language=[LaTeX]{TeX}
}
\begin{lstlisting}[frame=single]
% Included Package
\usepackage{float}
\usepackage{subcaption}
\usepackage{caption}
\captionsetup[figure]{labelfont={bf},name={Fig.},labelsep=period}
% Code
\begin{figure}[H]
    \centering
    \captionsetup{width=.85\textwidth}
	\caption{\blindtext}
	\label{fig:vert}
	\begin{subfigure}[t]{0.45\textwidth}
        % Sub-caption
        \centering
    		\caption{Contours of indifference curves at given estimates}
    		\includegraphics[width=\linewidth]{example-image-a}
	\end{subfigure}
	\par\bigskip % this line generates two rows
	\begin{subfigure}[t]{0.35\textwidth}
	    % Sub-caption
	    \centering
    		\caption{Vary $\mu_{R}$}
    		\includegraphics[width=0.5\linewidth]{example-image-b}
	\end{subfigure}
	\par\bigskip
    \parbox[t]{0.90\textwidth}{\footnotesize \emph{Note:} \blindtext}\\
\end{figure}
\end{lstlisting}

\par\medskip
\begin{figure}[H]
    \centering
    \captionsetup{width=.85\textwidth}
	\caption{\blindtext}
	\label{fig:vert}
	\begin{subfigure}[t]{0.45\textwidth}
        % Sub-caption
        \centering
    		\caption{Contours of indifference curves at given estimates}
    		\includegraphics[width=\linewidth]{example-image-a}
	\end{subfigure}
	\par\bigskip % this line generates two rows
	\begin{subfigure}[t]{0.35\textwidth}
	    % Sub-caption
	    \centering
    		\caption{Vary $\mu_{R}$}
    		\includegraphics[width=0.5\linewidth]{example-image-b}
	\end{subfigure}
	\par\bigskip
    \parbox[t]{0.90\textwidth}{\footnotesize \emph{Note:} \blindtext}\\
\end{figure}
\pagebreak

\section{Horizontal center align with subfigures}

The figure has the following features:
\begin{itemize}
    \item Two subfigures on the same line, possibly exceeding pagewidth
    \item Long wrapped captions
\end{itemize}

\lstset{%
  basicstyle=\footnotesize\ttfamily,
  breaklines=true,
  language=[LaTeX]{TeX}
}
\begin{lstlisting}[frame=single]
% Included Package
\usepackage{float}
\usepackage{subcaption}
\usepackage{caption}
% Code
\begin{figure}[H]
    \centering
	\label{fig:horz}
    \begin{subfigure}[t]{.51\textwidth}
        \centering
        % included graphics appear smaller than full
        \includegraphics[width=0.48\linewidth]{example-image-b}
        \caption{\blindtext}
    \end{subfigure}~
    % tilde makes subfigures appear on the same row even if > 1.0 of textwidth
	\begin{subfigure}[t]{0.48\textwidth}
	    \centering
	    % linewidth so that included graphics fills subfigure
    	\includegraphics[width=\linewidth]{example-image-a}
    	\caption{Vary $\mu_{R}$}
	\end{subfigure}
	\caption{\blindtext}
\end{figure}
\end{lstlisting}

\par\medskip
\begin{figure}[H]
    \centering
	\label{fig:horz}
    \begin{subfigure}[t]{.51\textwidth}
        \centering
        % included graphics appear smaller than full
        \includegraphics[width=0.48\linewidth]{example-image-b}
        \caption{\blindtext}
    \end{subfigure}~
    % tilde makes subfigures appear on the same row even if > 1.0 of textwidth
	\begin{subfigure}[t]{0.48\textwidth}
	    \centering
	    % linewidth so that included graphics fills subfigure
    	\includegraphics[width=\linewidth]{example-image-a}
    	\caption{Vary $\mu_{R}$}
	\end{subfigure}
	\caption{\blindtext}
\end{figure}
\pagebreak

\section{Horizontal and vertical center align with subfigures}
The figure has the following features:
\begin{itemize}
    \item Center align multiple rows of figures
    \item Caption width and wrapping same as subfigure width
\end{itemize}

\lstset{%
  basicstyle=\footnotesize\ttfamily,
  breaklines=true,
  language=[LaTeX]{TeX}
}
\begin{lstlisting}[frame=single]
% Included Package
\usepackage{float}
\usepackage{subcaption}
\usepackage{caption}
% Code
\newcommand{\subFigWidth}{0.27}
\newcommand{\subFigGraphWidth}{0.90}
\newcommand{\maincapwidth}{0.80}
\begin{figure}[H]
    \centering
	\label{fig:verthorz}
    \begin{subfigure}[t]{\subFigWidth\textwidth}
        \centering
        \includegraphics[width=\subFigGraphWidth\linewidth]{example-image-a}
        \captionsetup{width=\subFigGraphWidth\textwidth}
        \caption{$\alpha$}
    \end{subfigure}~
	\begin{subfigure}[t]{\subFigWidth\textwidth}
	    \centering
    	\includegraphics[width=\subFigGraphWidth\linewidth]{example-image-b}
    	\captionsetup{width=\subFigGraphWidth\textwidth}
    	\caption{Primary school consolidation--the closure of small community schools or their mergers into larger.}
	\end{subfigure}
	\par\medskip
	\begin{subfigure}[t]{\subFigWidth\textwidth}
        \centering
        \includegraphics[width=\subFigGraphWidth\linewidth]{example-image-a}
        \captionsetup{width=\subFigGraphWidth\textwidth}
        \caption{We consider heterogeneous treatment effects across groups defined at the intersections of minority status.}
    \end{subfigure}~
	\begin{subfigure}[t]{\subFigWidth\textwidth}
	    \centering
	    \captionsetup{width=\subFigGraphWidth\textwidth}
    	\includegraphics[width=\subFigGraphWidth\linewidth]{example-image-b}
    	\caption{Compared to villages with schools, villages whose schools had closed reported that the schools students}
	\end{subfigure}
	\captionsetup{width=\maincapwidth\textwidth}
	\caption{Results show that intersections of minority status, gender, and community characteristics can delineate significant heterogeneities in policy impacts.}
\end{figure}
\end{lstlisting}

\par\medskip
\newcommand{\subFigWidth}{0.27}
\newcommand{\subFigGraphWidth}{0.90}
\newcommand{\maincapwidth}{0.80}
\begin{figure}[H]
    \centering
	\label{fig:verthorz}
    \begin{subfigure}[t]{\subFigWidth\textwidth}
        \centering
        \includegraphics[width=\subFigGraphWidth\linewidth]{example-image-a}
        \captionsetup{width=\subFigGraphWidth\textwidth}
        \caption{$\alpha$}
    \end{subfigure}~
	\begin{subfigure}[t]{\subFigWidth\textwidth}
	    \centering
    	\includegraphics[width=\subFigGraphWidth\linewidth]{example-image-b}
    	\captionsetup{width=\subFigGraphWidth\textwidth}
    	\caption{Primary school consolidation--the closure of small community schools or their mergers into larger.}
	\end{subfigure}
	\par\medskip
	\begin{subfigure}[t]{\subFigWidth\textwidth}
        \centering
        \includegraphics[width=\subFigGraphWidth\linewidth]{example-image-a}
        \captionsetup{width=\subFigGraphWidth\textwidth}
        \caption{We consider heterogeneous treatment effects across groups defined at the intersections of minority status.}
    \end{subfigure}~
	\begin{subfigure}[t]{\subFigWidth\textwidth}
	    \centering
	    \captionsetup{width=\subFigGraphWidth\textwidth}
    	\includegraphics[width=\subFigGraphWidth\linewidth]{example-image-b}
    	\caption{Compared to villages with schools, villages whose schools had closed reported that the schools students}
	\end{subfigure}
	\captionsetup{width=\maincapwidth\textwidth}
	\caption{Results show that intersections of minority status, gender, and community characteristics can delineate significant heterogeneities in policy impacts.}
\end{figure}
\pagebreak
\end{document}

