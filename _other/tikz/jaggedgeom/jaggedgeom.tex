\documentclass{article}
\usepackage{amsmath} % aligned env

\usepackage{relsize} % make some math large
\usepackage{bm} % make some math bold

\usepackage{caption}
\usepackage{geometry}
\geometry{
	a4paper,
	noheadfoot=true,
	left=1.0in,
	right=1.0in,
	top=1.0in,
	bottom=1.0in,
}

% Specific Packages
\usepackage{float}
\usepackage{subcaption}
\usepackage{caption}
\captionsetup[figure]{labelfont={bf},name={Fig.},labelsep=period}

\usepackage{tikz}
\usetikzlibrary{decorations.pathreplacing}

\definecolor{myLightGray}{RGB}{191,191,191}
\definecolor{myGray}{RGB}{160,160,160}
\definecolor{myDarkGray}{RGB}{144,144,144}
\definecolor{myDarkRed}{RGB}{167,114,115}
\definecolor{myRed}{RGB}{255,58,70}
\definecolor{myGreen}{RGB}{0,255,71}

\begin{document}

\title{Latex Tikz Examples, Bridge Loan Charts}
\author{\href{https://fanwangecon.github.io/}{Fan Wang}\thanks{https://fanwangecon.github.io, repository: \href{https://fanwangecon.github.io/Tex4Econ/}{Tex4Econ}}}
\date{\today}

\section{Bridge construction}



% type a bridge
% x positions for a, b, c
\def\fatabxla{0.10*\textwidth}
\def\fatabxha{0.45*\textwidth}
\def\fatabxlb{0.35*\textwidth}
\def\fatabxhb{0.65*\textwidth}
\def\fatabxlc{0.55*\textwidth}
\def\fatabxhc{0.90*\textwidth}
% y positions for a, b, and c
\def\fatabyla{0.00}
\def\fatabyha{0.04*\textheight}
\def\fatabylb{0.01*\textheight}
\def\fatabyhb{0.05*\textheight}
\def\fatabylc{0.00}
\def\fatabyhc{0.04*\textheight}

% type b bridge
% x positions for a, b, c
\def\fatbbxla{0.10*\textwidth}
\def\fatbbxha{0.40*\textwidth}
\def\fatbbxlb{0.39*\textwidth}
\def\fatbbxhb{0.61*\textwidth}
\def\fatbbxlc{0.60*\textwidth}
\def\fatbbxhc{0.90*\textwidth}
% y positions for a, b, and c
\def\fatbbyla{0.00}
\def\fatbbyha{0.04*\textheight}
\def\fatbbylb{0.01*\textheight}
\def\fatbbyhb{0.05*\textheight}
\def\fatbbylc{0.00}
\def\fatbbyhc{0.04*\textheight}

% type c bridge
% x positions for a, b, c
\def\fatcbxla{0.10*\textwidth}
\def\fatcbxha{0.49*\textwidth}
\def\fatcbxlb{0.50*\textwidth}
\def\fatcbxhb{0.50*\textwidth}
\def\fatcbxlc{0.51*\textwidth}
\def\fatcbxhc{0.90*\textwidth}
% y positions for a, b, and c
\def\fatcbyla{0.00}
\def\fatcbyha{0.04*\textheight}
\def\fatcbylb{0.01*\textheight}
\def\fatcbyhb{0.05*\textheight}
\def\fatcbylc{0.00}
\def\fatcbyhc{0.04*\textheight}

\newcommand{\fatikzbridgetypea}{        
    \draw[line width=1mm, red] (\fatabxla,   \fatabyha) -- (\fatabxla, \fatabyla) -- (\fatabxha, \fatabyla) -- (\fatabxha, \fatabyha) ;
    \draw[line width=1mm, brown] (\fatabxlc,   \fatabyhc) -- (\fatabxlc, \fatabylc) -- (\fatabxhc, \fatabylc) -- (\fatabxhc, \fatabyhc) ;
    \draw[line width=1mm, blue] (\fatabxlb,   \fatabylb) -- (\fatabxlb, \fatabyhb) -- (\fatabxhb, \fatabyhb) -- (\fatabxhb, \fatabylb) ;
}
\newcommand{\fatikzbridgetypeb}{
    \draw[line width=1mm, red] (\fatbbxla,   \fatbbyha) -- (\fatbbxla, \fatbbyla) -- (\fatbbxha, \fatbbyla) -- (\fatbbxha, \fatbbyha) ;
    \draw[line width=1mm, brown] (\fatbbxlc,   \fatbbyhc) -- (\fatbbxlc, \fatbbylc) -- (\fatbbxhc, \fatbbylc) -- (\fatbbxhc, \fatbbyhc) ;
    \draw[line width=1mm, blue] (\fatbbxlb,   \fatbbylb) -- (\fatbbxlb, \fatbbyhb) -- (\fatbbxhb, \fatbbyhb) -- (\fatbbxhb, \fatbbylb) ;
}
\newcommand{\fatikzbridgetypec}{
    \draw[line width=1mm, red] (\fatcbxla,   \fatcbyha) -- (\fatcbxla, \fatcbyla) -- (\fatcbxha, \fatcbyla) -- (\fatcbxha, \fatcbyha) ;
    \draw[line width=1mm, brown] (\fatcbxlc,   \fatcbyhc) -- (\fatcbxlc, \fatcbylc) -- (\fatcbxhc, \fatcbylc) -- (\fatcbxhc, \fatcbyhc) ;
    \draw[line width=1mm, blue] (\fatcbxlb,   \fatcbylb) -- (\fatcbxlb, \fatcbyhb) -- (\fatcbxhb, \fatcbyhb) -- (\fatcbxhb, \fatcbylb) ;
}

\begin{figure}[H]
\centering
\begin{subfigure}[t]{.33\textwidth}
    \centering
    \begin{tikzpicture}[scale=1.0, every node/.style={scale=0.5}]]
        \fatikzbridgetypea
    \end{tikzpicture}
    \caption{abc}
\end{subfigure}~
% tilde makes subfigures appear on the same row even if > 1.0 of textwidth
\begin{subfigure}[t]{0.33\textwidth}
    \centering
    \begin{tikzpicture}[scale=1.0, every node/.style={scale=0.5}]]
        \fatikzbridgetypeb
    \end{tikzpicture}
    \caption{Vary $\mu_{R}$}
\end{subfigure}~
\begin{subfigure}[t]{0.33\textwidth}
    \centering
    \begin{tikzpicture}[scale=1.0, every node/.style={scale=0.5}]]
        \fatikzbridgetypec`
    \end{tikzpicture}
    \caption{Vary $\mu_{R}$}
\end{subfigure}
\caption{efg}
\label{fig:hortwo}
\end{figure}

\subsection{Three bridges annotated}


% type a bridge
% x positions for a, b, c
\def\tabxla{0.10*\textwidth}
\def\tabxha{0.45*\textwidth}
\def\tabxlb{0.35*\textwidth}
\def\tabxhb{0.65*\textwidth}
\def\tabxlc{0.55*\textwidth}
\def\tabxhc{0.90*\textwidth}
% text positions along x
\def\tatxtxa{0.275*\textwidth}
\def\tatxtxb{0.500*\textwidth}
\def\tatxtxc{0.725*\textwidth}
% y positions for a, b, and c
\def\tabyla{0.00}
\def\tabyha{0.04*\textheight}
\def\tabylb{0.01*\textheight}
\def\tabyhb{0.05*\textheight}
\def\tabylc{0.00}
\def\tabyhc{0.04*\textheight}
% text positions along y
\def\tatytxa{-0.015*\textheight}
\def\tatytxb{0.065*\textheight}
\def\tatytxc{-0.015*\textheight}

% type b bridge
% x positions for a, b, c
\def\tbbxla{0.10*\textwidth}
\def\tbbxha{0.40*\textwidth}
\def\tbbxlb{0.39*\textwidth}
\def\tbbxhb{0.61*\textwidth}
\def\tbbxlc{0.60*\textwidth}
\def\tbbxhc{0.90*\textwidth}
% text positions along x
\def\tbtxtxa{0.25*\textwidth}
\def\tbtxtxb{0.500*\textwidth}
\def\tbtxtxc{0.75*\textwidth}
% y positions for a, b, and c
\def\tbbyla{0.00}
\def\tbbyha{0.04*\textheight}
\def\tbbylb{0.01*\textheight}
\def\tbbyhb{0.05*\textheight}
\def\tbbylc{0.00}
\def\tbbyhc{0.04*\textheight}
% text positions along y
\def\tbtytxa{-0.015*\textheight}
\def\tbtytxb{0.065*\textheight}
\def\tbtytxc{-0.015*\textheight}

% type c bridge
% x positions for a, b, c
\def\tcbxla{0.10*\textwidth}
\def\tcbxha{0.49*\textwidth}
\def\tcbxlb{0.50*\textwidth}
\def\tcbxhb{0.50*\textwidth}
\def\tcbxlc{0.51*\textwidth}
\def\tcbxhc{0.90*\textwidth}
% text positions along x
\def\tctxtxa{0.295*\textwidth}
\def\tctxtxb{0.500*\textwidth}
\def\tctxtxc{0.705*\textwidth}
% y positions for a, b, and c
\def\tcbyla{0.00}
\def\tcbyha{0.04*\textheight}
\def\tcbylb{0.01*\textheight}
\def\tcbyhb{0.05*\textheight}
\def\tcbylc{0.00}
\def\tcbyhc{0.04*\textheight}
% text positions along y
\def\tctytxa{-0.015*\textheight}
\def\tctytxb{0.065*\textheight}
\def\tctytxc{-0.015*\textheight}

\newcommand{\tikzbridgetypea}{        
    \draw[line width=1mm, red] (\tabxla,   \tabyha) -- (\tabxla, \tabyla) -- (\tabxha, \tabyla) -- (\tabxha, \tabyha) ;
    \draw[dash pattern={on 10pt off 3pt on 10pt off 3pt}, line width=1mm, brown] (\tabxlc,   \tabyhc) -- (\tabxlc, \tabylc) -- (\tabxhc, \tabylc) -- (\tabxhc, \tabyhc) ;
    \draw[densely dotted, line width=1mm, blue] (\tabxlb,   \tabylb) -- (\tabxlb, \tabyhb) -- (\tabxhb, \tabyhb) -- (\tabxhb, \tabylb) ;

    \node[align=center] (A) at (\tatxtxa, \tatytxa) {$\bm{\tilde{l}}$ };
    \node[align=center] (B) at (\tatxtxb, \tatytxb) {$\bm{\widehat{l}}$ };
    \node[align=center] (C) at (\tatxtxc, \tatytxc) {$\bm{l}$ };
}
\newcommand{\tikzbridgetypeb}{
    \draw[line width=1mm, red] (\tbbxla,   \tbbyha) -- (\tbbxla, \tbbyla) -- (\tbbxha, \tbbyla) -- (\tbbxha, \tbbyha) ;
    \draw[dash pattern={on 10pt off 3pt on 10pt off 3pt}, line width=1mm, brown] (\tbbxlc,   \tbbyhc) -- (\tbbxlc, \tbbylc) -- (\tbbxhc, \tbbylc) -- (\tbbxhc, \tbbyhc) ;
    \draw[densely dotted, line width=1mm, blue] (\tbbxlb,   \tbbylb) -- (\tbbxlb, \tbbyhb) -- (\tbbxhb, \tbbyhb) -- (\tbbxhb, \tbbylb) ;

    \node[align=center] (A) at (\tbtxtxa, \tbtytxa) {$\bm{\tilde{l}}$ };
    \node[align=center] (B) at (\tbtxtxb, \tbtytxb) {$\bm{\widehat{l}}$ };
    \node[align=center] (C) at (\tbtxtxc, \tbtytxc) {$\bm{l}$ };    
}
\newcommand{\tikzbridgetypec}{
    \draw[line width=1mm, red] (\tcbxla,   \tcbyha) -- (\tcbxla, \tcbyla) -- (\tcbxha, \tcbyla) -- (\tcbxha, \tcbyha) ;
    \draw[dash pattern={on 10pt off 3pt on 10pt off 3pt}, line width=1mm, brown] (\tcbxlc,   \tcbyhc) -- (\tcbxlc, \tcbylc) -- (\tcbxhc, \tcbylc) -- (\tcbxhc, \tcbyhc) ;
    \draw[densely dotted, line width=1mm, blue] (\tcbxlb,   \tcbylb) -- (\tcbxlb, \tcbyhb) -- (\tcbxhb, \tcbyhb) -- (\tcbxhb, \tcbylb) ;
    
    \node[align=center] (A) at (\tctxtxa, \tctytxa) {$\bm{\tilde{l}}$ };
    \node[align=center] (B) at (\tctxtxb, \tctytxb) {$\bm{\widehat{l}}$ };
    \node[align=center] (C) at (\tctxtxc, \tctytxc) {$\bm{l}$ };        
}
\begin{figure}[H]
\centering
\begin{subfigure}[t]{.30\textwidth}
    \centering
    \begin{tikzpicture}[scale=1.0, every node/.style={scale=1}]]
        \tikzbridgetypea
    \end{tikzpicture}
    \caption{Loan $\tilde{l}$ from loan set $A$ (solid red), loan $\widehat{l}$ from loan set $B$ (dotted blue), and loan $l$ from loan set $C$ (dashed brown) are ``hooked'' to each other, with some gaps between the hooks---the bridging loan $\widehat{l}$ is taken out before the due date of loan $\tilde{l}$ and repaid after the origination date of loan $l$.}
\end{subfigure}
\hfill
\begin{subfigure}[t]{0.30\textwidth}
    \centering
    \begin{tikzpicture}[scale=1.0, every node/.style={scale=1}]]
        \tikzbridgetypeb
    \end{tikzpicture}
    \caption{The ``hooks'' between loan set loan $\tilde{l}$ from loan set $A$ (solid red), loan $\widehat{l}$ from loan set $B$ (dotted blue), and loan $l$ from loan set $C$ (dashed brown) are tight---the bridging loan $\widehat{l}$ is taken out in the same month loan $\tilde{l}$ is due and repaid in the same month that loan $\widehat{l}$ originates.}
\end{subfigure}
\hfill
\begin{subfigure}[t]{0.30\textwidth}
    \centering
    \begin{tikzpicture}[scale=1.0, every node/.style={scale=1}]]
        \tikzbridgetypec`
    \end{tikzpicture}
    \caption{The ``bridge'' is length-less---the bridge loan $\widehat{l}$ from loan set $B$ (dotted blue) originates and is repaid in the same month that loan $\tilde{l}$ is repaid and loan $l$ originates.}
\end{subfigure}
\captionsetup{width=1\textwidth}
\caption{
The bridge-loan linkage defined in Eq. \eqref{eq:rills:invest:loan:set:bridge} allows for three types of  triply-linked bridge loans, $\left(\tilde{l}, \widehat{l}, l\right) \in \text{BRIDGE}_{j,k}$. We illustrate the three types in three panels. Investment-linked loan $\tilde{l}$ (set $A$) is shown in red, middle bridge-loan $\widehat{l}$ (set $B$) is shown in blue, and third loan $l$ (set $C$) is shown in brown. The left and right vertical lines for each loan illustrates origination and due months for each loan.
}
\label{fig:tikz:bridge:types}
\end{figure}



\subsection{Empty Frame}

% we need to define three boxes
% xl, xh, yl, yh, for a, b, and c
% x positions for a, b, and c
\def\bxla{0.02*\textwidth}
\def\bxha{0.12*\textwidth}
\def\bxlb{0.10*\textwidth}
\def\bxhb{0.20*\textwidth}
\def\bxlc{0.18*\textwidth}
\def\bxhc{0.28*\textwidth}
% y positions for a, b, and c
\def\byla{0.00}
\def\byha{0.03*\textheight}
\def\bylb{0.02*\textheight}
\def\byhb{0.05*\textheight}
\def\bylc{0.00}
\def\byhc{0.03*\textheight}

\def\fgw{1.0*\textwidth}
\def\fgh{0.20*\textheight}

\newcommand{\tikzframe}{        
	\draw (0,     \fgh) -- (\fgw,     \fgh);
	\draw[loosely dotted] (0, 0.5*\fgh) -- (\fgw, 0.5*\fgh);
        \draw (0, 0       ) -- (\fgw, 0       );
        
	% \draw[line width=0.25mm, red] (\bxla, \byha) -- (\bxla, \byla);
        \draw[line width=1mm, red] (\bxla,   \byha) -- (\bxla, \byla) -- (\bxha, \byla) -- (\bxha, \byha) ;
        \draw[line width=1mm, blue] (\bxlb,   \bylb) -- (\bxlb, \byhb) -- (\bxhb, \byhb) -- (\bxhb, \bylb) ;
        \draw[line width=1mm, brown] (\bxlc,   \byhc) -- (\bxlc, \bylc) -- (\bxhc, \bylc) -- (\bxhc, \byhc) ;
        
        % \draw (\aax, 0) -- (\aax+\aaw, 0)
	% Three Straight Lines
	\draw[dashed] (0.125*\fgw, 0  ) -- (0.125*\fgw,   \fgh);
	\draw[dotted] (0.500*\fgw, 0.2*\fgh ) -- (0.500*\fgw,   0.8*\fgh);
	\draw[dashed] (0.875*\fgw, 0  ) -- (0.875*\fgw,   \fgh);

	\node[align=center] at (0.065*\fgw, 0.900*\fgh) {last\\period};
	\node[align=center] at (0.500*\fgw, 0.900*\fgh) {current period};
	\node[align=center] at (0.935*\fgw, 0.900*\fgh) {next\\period};

	\node[align=center] at (0.325*\fgw, 0.100*\fgh) {shocks and \\ income realization};
	\node[align=center] at (0.675*\fgw, 0.100*\fgh) {asset choices};
}
\newcommand{\tikzframebelowtext}[4]{
  \def\fgiw{#1*\textwidth}
  \def\fgih{#2*\textheight}
  \def\fgiwm{#3*\fgiw}
  \def\fgihm{#4*\fgih}

  % left pane middle point, align horizontally middle pane
  \def\fgiwlm{0.125*\textwidth + \fgiwm*0.5 - 0.5*0.125*\textwidth}
  % right pane middle point, align horizontally middle in pane
  \def\fgiwrm{\fgiwm + \fgiw*0.5 - 0.125*\textwidth*0.5 - \fgiwm*0.5}

  % fgihtb: figure internal height top pane bottom
  % bottom 10 percent space of top pane space
  \def\fgihtb{\fgihm + \fgih*0.085 - \fgihm*0.085}


  % There is a flow line on top, limited text
  % three parts, left prior, right after, middle current.
  % current in two parts, shock realization and choices.
  % In top part, span charts, simple text
  % in bottom part, sufficient height to allow for detailed text
  % descriptions.

  % Bottom and top lines
  \draw [solid] (0, \fgih) -- (\fgiw, \fgih);
  \draw [solid] (0, 0    ) -- (\fgiw, 0    );

  % Left and right lines
  \draw [solid] (0,     0) -- (0,     \fgih);
  \draw [solid] (\fgiw, 0) -- (\fgiw, \fgih);

  % verticle middle line
  \draw [solid] (\fgiwm, 0) -- (\fgiwm,   \fgih);
  % horizontal middle line (divide text and flow)
  \draw [solid] (0, \fgihm) -- (\fgiw,   \fgihm);

  % left border last period
  \draw[dashed] (0.125*\fgiw, 0  ) -- (0.125*\fgiw,   \fgih);
  % right border next period
  \draw[dashed] (0.875*\fgiw, 0  ) -- (0.875*\fgiw,   \fgih);

  % %
  % \draw[loosely dotted] (0, 0.575*\fgih) -- (\fgiw, 0.575*\fgih);
  % \draw[loosely dotted] (0, 0.350*\fgih) -- (\fgiw, 0.350*\fgih);

  \node[align=center] at (0.065*\fgiw, 0.900*\fgih) {last\\period};
  \node[align=center] at (0.500*\fgiw, 0.900*\fgih) {current period};
  \node[align=center] at (0.935*\fgiw, 0.900*\fgih) {next\\period};

  \node[align=center] at (\fgiwlm, \fgihtb) {shocks};
  \node[align=center] at (\fgiwrm, \fgihtb) {asset choices};
}


100 percent frame with 20 percent page height and 100 percent page width:

\begin{center}
\begin{tikzpicture}[scale=1.0]
\tikzframe
\end{tikzpicture}
\end{center}

50 percent rescaled center-aligned frame with 20 percent page height and 100 percent page width:

\begin{center}
\begin{tikzpicture}[scale=0.5, every node/.style={scale=0.5}]]
\tikzframe
\end{tikzpicture}
\end{center}

\subsection{Frame Filled Choice}

%%%%%%%%%%%%%%%%%%%%%%%%%%%%%%%%%%%%%%%%%%%%%%%%%%%%%%%%%%%
%%% Define Common Blocks
%%%%%%%%%%%%%%%%%%%%%%%%%%%%%%%%%%%%%%%%%%%%%%%%%%%%%%%%%%%
% Define text block
\newcommand{\textblock}{
\node[align=center] (A) at (0.065*\fgw, 0.5*\fgh)
    {choose \\ $k, b$ };
\node[align=center] (B) at (0.370*\fgw, 0.5*\fgh)
    {$\Gamma\left(k,b,\epsilon\right)$ };
\node[align=center] (C) at (0.790*\fgw, 0.5*\fgh)
    {$\begin{aligned}
      c &= \Gamma\cdot\phi\\
      k &= \Gamma\cdot\phi\cdot\theta\\
      b &= \Gamma\cdot\phi\cdot\left(1-\theta\right)
      \end{aligned}$};
\node[align=center] (D) at (0.935*\fgw, 0.5*\fgh)
    {draw \\$\epsilon^{\prime}$ shock};
}

% Define continuous choices
\def\famm{10mm}
\tikzstyle{level 1}=[level distance=\famm,sibling distance=\famm]
\tikzstyle{level 2}=[level distance=\famm,sibling distance=\famm]
\newcommand{\shkspan}[2]{
\node(0)[solid node, align=center, % solid node gives fist (left to right) black dot
label=left:{
   $\textcolor{black}{\epsilon}$
    }] at (#1, #2) {}
[grow=right]
child{
    [black] node(31)[]{}
    edge from parent
          node[sloped, below, black]{}
}
child{
    edge from parent
      [draw=none]
  }
child{
    [red] node(33)[]{}
    edge from parent
          node[sloped, above, black]{}
};
\draw[dashed, bend right]
(31) to (33);
}
\newcommand{\ctsspans}[2]{
\node[solid node, align=center, % solid node gives fist (left to right) black dot
label=left:{
   $\textcolor{black}{\phi}$
    }] at (#1, #2) {}
[grow=right]
child{
[black] node(11)[]{}
edge from parent
      node[sloped, below, black]{$\phi=0$}
}
child{
[grow=right]
% The starting node for theta
node(12)[solid node, xshift=9, yshift=0, % solid node gives second (left to right) black dot
                label=left:{
                  $\theta$
                }]{}
% Draw the theta branches
    child{
        [red] node(21)[]{}
        edge from parent
            node[sloped, below, black]{$\theta=0$}
    }
    child{
        [white] node(22)[xshift=16, yshift=0]{}
        edge from parent
          [draw=none]
    }
    child{
        [black] node(23)[]{}
        edge from parent
            node[sloped, above, black]{$\theta=1$}
    }
    edge from parent
      [draw=none]
  }
child{
[red] node(13)[]{}
edge from parent
      node[sloped, above, black]{$\phi=1$}
};
\draw[dashed, bend right]
(11) to (13);
\draw[dashed, bend right]
(21) to (23);
}


\subsubsection{Spans to Left of Descriptions, 100 percent}
% A. Frame
\begin{center}
\begin{tikzpicture}[scale=1.0]
\tikzframe
% B. Text
\textblock
% C. Arrows
\draw [->, line width=0.25mm] (0.115*\fgw, 0.5*\fgh) -- (0.195*\fgw, 0.5*\fgh);
\draw [->, line width=0.25mm] (0.410*\fgw, 0.5*\fgh) -- (0.480*\fgw, 0.5*\fgh);
\draw [->, line width=0.25mm] (0.835*\fgw, 0.5*\fgh) -- (0.890*\fgw, 0.5*\fgh);
% D. Span Children
\ctsspans{0.530*\fgw}{0.500*\fgh}
\shkspan{0.235*\fgw}{0.500*\fgh}
% \draw  -- (10,0);
\end{tikzpicture}
\end{center}

\subsubsection{Spans to Left of Descriptions, 50 percent}
% A. Frame
\begin{center}
\begin{tikzpicture}[scale=0.5, every node/.style={scale=0.5}]
\tikzframe
% B. Text
\textblock
% C. Arrows
\draw [->, line width=0.25mm] (0.115*\fgw, 0.5*\fgh) -- (0.195*\fgw, 0.5*\fgh);
\draw [->, line width=0.25mm] (0.410*\fgw, 0.5*\fgh) -- (0.480*\fgw, 0.5*\fgh);
\draw [->, line width=0.25mm] (0.835*\fgw, 0.5*\fgh) -- (0.890*\fgw, 0.5*\fgh);
% D. Span Children
\ctsspans{0.530*\fgw}{0.500*\fgh}
\shkspan{0.235*\fgw}{0.500*\fgh}
% \draw  -- (10,0);
\end{tikzpicture}
\end{center}

\subsubsection{Frame Filled Choice Spans Above Descriptions}
% A. Frame
\begin{center}
\begin{tikzpicture}[scale=1.0, every node/.style={scale=1.0}]
\tikzframebelowtext{1}{0.25}{0.4}{0.35}
% \draw  -- (10,0);
\end{tikzpicture}
\end{center}

\begin{center}
\begin{tikzpicture}[scale=1.0, every node/.style={scale=1.0}]
\tikzframebelowtext{0.6}{0.20}{0.4}{0.35}
% \draw  -- (10,0);
\end{tikzpicture}
\end{center}

\begin{center}
\begin{tikzpicture}[scale=1.0, every node/.style={scale=1.0}]
\tikzframebelowtext{1}{0.15}{0.5}{0.5}
% \draw  -- (10,0);
\end{tikzpicture}
\end{center}

\subsubsection{Frame Filled Choice Spans Above Descriptions}
% A. Frame
\begin{center}
\begin{tikzpicture}[scale=1.0, every node/.style={scale=1.0}]
\tikzframebelowtext{1}{0.25}{0.4}{0.4}
% B. Text
\node[align=center] (A) at (0.065*\fgw, 0.65*\fgh)
    {choose \\ $k, b$ };
\node[align=center] (B) at (0.320*\fgw, 0.350*\fgh)
    {$\Gamma\left(k,b,\epsilon\right)$ };
\node[align=center] (C) at (0.675*\fgw, 0.350*\fgh)
    {$\begin{aligned}
      c &= \Gamma\cdot\phi\\
      k &= \Gamma\cdot\phi\cdot\theta\\
      b &= \Gamma\cdot\phi\cdot\left(1-\theta\right)
      \end{aligned}$};
\node[align=center] (D) at (0.935*\fgw, 0.65*\fgh)
    {draw \\$\epsilon^{\prime}$ shock};
% C. Arrows
\draw [->, line width=0.25mm] (0.115*\fgw, 0.65*\fgh) -- (0.250*\fgw, 0.65*\fgh);
\draw [->, line width=0.25mm] (0.410*\fgw, 0.65*\fgh) -- (0.550*\fgw, 0.65*\fgh);
\draw [->, line width=0.25mm] (0.790*\fgw, 0.65*\fgh) -- (0.890*\fgw, 0.65*\fgh);
% D. Span Children
\ctsspans{0.600*\fgw}{0.650*\fgh}
\shkspan{0.300*\fgw}{0.650*\fgh}
% \draw  -- (10,0);
\end{tikzpicture}
\end{center}


\end{document}
