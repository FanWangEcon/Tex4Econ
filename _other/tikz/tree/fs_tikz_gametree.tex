\documentclass{article}

\usepackage{blindtext}
\usepackage{multicol}
\usepackage{caption}
\usepackage{amsmath}
\usepackage{tikz}
\usepackage{pgfplots}
\usepackage{hyperref}
\hypersetup{
    colorlinks=true,
    linkcolor=blue,
    filecolor=magenta,
    urlcolor=cyan,
}
\usepackage{geometry}
\geometry{
	a4paper,
	noheadfoot=true,
	left=1.0in,
	right=1.0in,
	top=1.0in,
	bottom=1.0in
}

% url package
\usepackage{hyperref}
\usepackage{subcaption}

% Titling and Author
\title{Latex Tikz Examples, Discrete and Continuous Strategy Trees}
\author{\href{https://fanwangecon.github.io/}{Fan Wang}\thanks{https://fanwangecon.github.io, repository: \href{https://fanwangecon.github.io/Tex4Econ/}{Tex4Econ}}}
\date{\today}

\begin{document}

\maketitle

% A.1 Node styles
\tikzset{
    % Two node styles for game trees: solid and hollow
    solid node/.style={circle,draw,inner sep=1.5,fill=black},
    hollow node/.style={circle,draw,inner sep=1.5,fill=white}
}

\section{Grow Tree towards Different Directions}

\begin{center}
\begin{tikzpicture}[scale=1.5,font=\footnotesize]
\tikzstyle{level 1}=[level distance=10mm,sibling distance=10mm]
\tikzstyle{level 2}=[level distance=12mm,sibling distance=7mm]
\tikzstyle{level 3}=[level distance=6mm,sibling distance=6mm]
\tikzstyle{level 4}=[level distance=4mm,sibling distance=4mm]
% B Level 1 point
\node(0)[solid node,label=left:{$P1$}]{}
[grow=right]
child{
    [red]
    node(1)[solid node, label=below:{red dot}]{}
    }
child{
    [purple]
    node(2)[solid node]{}
    [grow=right]
    child
    child{
        [black]
        node(4)[hollow node, green]{}
        [grow=right]
        child
        child{
            [grow=right]
            child
            child
            child
            }
        }
    child
    child
    child{
            [black]
            node(5)[hollow node, red]{}
            [grow=up]
            child
            child
        }
    }
child{
    [black]
    node(3)[solid node, label=above:{hi there}, green]{}
    };
\end{tikzpicture}
\end{center}

\section{Continuous and Discrete}

\begin{center}
\begin{tikzpicture}[scale=1.5,font=\footnotesize]

% A.2 Level Styles
    % Specify spacing for each level of the tree
    \tikzstyle{level 1}=[level distance=17mm,sibling distance=25mm]
    \tikzstyle{level 2}=[level distance=15mm,sibling distance=15mm]

% B Level 1 point
\node(0)[solid node,label=above:{$P1$}]{}

% C Level 2 Node (child nodes)
% C.1 First child node, left
child{
    % C.1.a Node line color
    [red]
    % C.1.b Node dot
    node(1)[solid node, label=below:{red dot}]{}
    % C.1.c now go to edge
    edge from parent
        % C.1.c Node along edge
        node[sloped, above, black, text width=3cm]{Minimum Choice showing up along this red line}
    }
% C.2 Middle Child node, do not show line, invisible
child{
    % Main node of level 2 middle node, same y-axis height as node left and right
    % Y-shift to move level 3 child node lower
    [purple] node(2)[solid node, xshift=30, yshift=-50, label=right:{hello}]{}
    % D.1 Left level 3 (from middle level 2)
    child{
        [black]
        node[hollow node,label=below:{$(a,b)$}]{}
        edge from parent
            node[left]{$C$}
        }
    child{
        [black]
        node[hollow node,label=below:{$(c,d)$}]{}
        edge from parent
            node[right]{$D$}
        }
    edge from parent
        %note that you need to adjust the yshift if you change the level distance
        node[left, black, xshift=-5, yshift=0, text width = 1mm]{$\alpha$\\$\beta$\\$\sigma$}
    }
child{
    [black]
    node(3)[solid node, label=right:{hi there}, green]{}
    edge from parent
        node[right, xshift=0, yshift=15, text width=2cm]{Some text for this edge}
    };
% information set
\draw[dashed, bend right]
    (1) to (3);
\draw[dashed, bend left, line width=0.5mm, blue]
    (0) to
        node[right, text width=1.25cm, xshift=5, yshift=-10]
            {\tiny{label right of blue dashed line}}
        (2);
\end{tikzpicture}
\end{center}

\section{Grow Three to the Right, Discrete and Continuous Muliple}

There are three choices, choose consumption share first, which also means the residual goes to overall investment share.

\begin{enumerate}
  \item \textbf{continous}, choose consumption, or aggregate investment, rightward bend span, three node, center longer to match bend
  \item \textbf{continous}, choose investment in capital, type 1 asset, residual goes to type two asset, rightward span, three node, center longer to match bend
  \item \textbf{discrete}, within type 2 asset, choose a achieve the desired level of type two, rightward, five nodes, equi-distance
\end{enumerate}

\subsection{Nodes Only}

When developing trees, to simply the problem, first just draw nodes and branches, no text, no benidng, no coloring.

\def\famm{5mm}
\begin{center}
\begin{tikzpicture}[scale=1.5,font=\footnotesize]W
\tikzstyle{level 1}=[level distance=\famm,sibling distance=\famm]
\tikzstyle{level 2}=[level distance=\famm,sibling distance=\famm]
\tikzstyle{level 3}=[level distance=\famm,sibling distance=\famm]
% B Level 1 point
\node(0)[solid node,label=left:{$P1$}]{}
[grow=right]
child
child{
    [grow=right]
    child
    child{
        [grow=right]
        child
        child
        child
        child
        child
        }
    child
    }
child;
\end{tikzpicture}
\end{center}




% \begin{istgame}[font=\scriptsize]
% \setistgrowdirection{east}
% \cntmdistance{20mm}{20mm}
% \cntmAistb{q_1=0}[at end,below]{q_1=1,000}[at end,above]
% \istrootcntmA(0){1}
%   \istbA[draw=none]
%   \endist
% \cntmAistb{q_2=0}[at end,below]{q_2=1,000}[at end,above]
% \istrootcntmA(1)(0-1)<[xshift=3pt]90>{2}
%   \istbA[draw=none]{}{\pi_1,\pi_2}
%   \endist
% \end{istgame}



\end{document}
