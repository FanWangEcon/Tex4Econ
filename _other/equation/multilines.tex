\documentclass[12pt,english]{article}
\usepackage[utf8]{inputenc}
\usepackage[T1]{fontenc}
\usepackage{mathtools}
\usepackage{relsize} % make some math large
\usepackage{bm} % make some math bold
\usepackage{bbm}
% url package

\usepackage{hyperref}
\hypersetup{
    colorlinks=true,
    linkcolor=blue,
    filecolor=magenta,
    urlcolor=cyan,
}

% Titling and Author
\title{Latex Example, Multiple Lines Equations}
\author{\href{https://fanwangecon.github.io/}{Fan Wang}\thanks{https://fanwangecon.github.io, repository: \href{https://fanwangecon.github.io/Tex4Econ/}{Tex4Econ}}}
\date{\today}

\begin{document}

\maketitle

\section{Indicator Function}
\begin{align}\label{eq:indi}
  \begin{gathered}
    \mathbbm{1}\left\{1 + 2 +3\right\}
    \\
    \mathlarger{\mathlarger{\mathlarger{
      \mathbbm{1}
    }}}
    \left\{
      \begin{array}{cc}
          1 + 2 + 3\\
          + 4 + 5\\
          > 0\\
      \end{array}
    \right\}
    \\
    \mathlarger{\mathlarger{\mathlarger{
    \mathlarger{\mathlarger{\mathlarger{
      \mathbbm{1}
    }}}}}}
    \left\{
      \begin{array}{cc}
          1 + 2 + 3\\
          + 4 + 5 + 6 \\
          + 7 + 8 + 9 \\
          > 0\\
      \end{array}
    \right\}
  \end{gathered}
\end{align}


\section{Multiple Lines of Equation}

Below, I demonstrate several possible ways of aligning and labeling four lines of equations. First, I label some of the four lines with own labels and align by equality sign. Second, we align by the equality sign but use one single label for all lines of equations. Third, we center align on page label each line using gather. Fourth, we center align on page with one label putting gather inside align.

%%%%%%%%%%%%%%%%%%%%%%%%%%%%%%%%%%%%%%%%%%%%%%%%%
%%% A.1 Align and Split
%%%%%%%%%%%%%%%%%%%%%%%%%%%%%%%%%%%%%%%%%%%%%%%%%
\pagebreak
\subsection{Align and Split}
\subsubsection{Align and Label Lines}
\begin{verbatim}
  \begin{align}
    x_1  =& 1        \label{eq:lineone}\\
    x_2  =& 1        \nonumber\\
    h(x) =& f(-20 + 15 + 17) \label{eq:linethree}\\
    h(x) =& f(12) \approx 1  \label{eq:linefour}
  \end{align}
\end{verbatim}
\begin{align}
  x_1  =& 1        \label{eq:lineone}\\
  x_2  =& 1        \nonumber\\
  h(x) =& f(-20 + 15 + 17) \label{eq:linethree}\\
  h(x) =& f(12) \approx 1  \label{eq:linefour}
\end{align}
\subsubsection{Align Split One Label}
\begin{verbatim}
\begin{align}\label{eq:split}
  \begin{split}
    x_1 &= 1 \\
    y &= 1        \\
    h(x) &= f(-20 + 10 + 20) \\
    g(x) &= f(12) \approx 2
  \end{split}
\end{align}
\end{verbatim}
\begin{align}\label{eq:split}
  \begin{split}
    x_1 &= 1 \\
    y &= 1        \\
    h(x) &= f(-20 + 10 + 20) \\
    g(x) &= f(12) \approx 2
  \end{split}
\end{align}

%%%%%%%%%%%%%%%%%%%%%%%%%%%%%%%%%%%%%%%%%%%%%%%%%
%%% A.2 Align and Gather
%%%%%%%%%%%%%%%%%%%%%%%%%%%%%%%%%%%%%%%%%%%%%%%%%
\pagebreak
\subsection{Align and Gather}
\subsubsection{Gather and Label Lines}
\begin{verbatim}
  \begin{gather}
    x_1  = 1        \label{eq:lineone}\\
    x_2  = 1        \nonumber\\
    h(x) = f(-20 + 15 + 17) \label{eq:linethree}\\
    h(x) = f(12) \approx 1  \label{eq:linefour}
\end{gather}
\end{verbatim}
\begin{gather}
  x_1  = 1        \label{eq:lineone}\\
  x_2  = 1        \nonumber\\
  h(x) = f(-20 + 15 + 17) \label{eq:linethree}\\
  h(x) = f(12) \approx 1  \label{eq:linefour}
\end{gather}
\subsubsection{Align Gathered}
\begin{verbatim}
  \label{eq:gathered}
  \begin{gathered}
    x_1 = 1       \\
    x_2 = 1        \\
    h(x) = f(-20 + 15 + 17) \\
    h(x) = f(12) \approx 1
  \end{gathered}
\end{verbatim}
\begin{align}
  \label{eq:gathered}
  \begin{gathered}
    x_1 = 1       \\
    x_2 = 1        \\
    h(x) = f(-20 + 15 + 17) \\
    h(x) = f(12) \approx 1
  \end{gathered}
\end{align}

%%%%%%%%%%%%%%%%%%%%%%%%%%%%%%%%%%%%%%%%%%%%%%%%%
%%% B. Substack vs Array
%%%%%%%%%%%%%%%%%%%%%%%%%%%%%%%%%%%%%%%%%%%%%%%%%
\pagebreak
\section{Substack vs Array}
\begin{verbatim}
\begin{align}
    \begin{split}
    \label{eq:Value}
    v_{ih}\left(a,z\right)
        =
        \max_{
            \substack{
                c>0\\
                a' \in \{0,[\bar{A}_{ih},\infty)\}
                }
            }
        u\left(c\right)+  \beta \int v_{ih}\left(a',z'\right)f(z'|z)dz'\\
    \label{eq:Value}
    v_{ih}\left(a,z\right)
        =
        \max_{
            \begin{array}{cc}
                c>0\\
                a' \in \{0,[\bar{A}_{ih},\infty)\}\\
            \end{array}
            }
        u\left(c\right)+  \beta \int v_{ih}\left(a',z'\right)f(z'|z)dz'\\
    \end{split}
\end{align}
\end{verbatim}

Using Substack, fonts are small under max:
\begin{align}
    \begin{split}
        \label{eq:Value}
        v_{ih}\left(a,z\right)
        =
        \max_{
            \substack{
                c>0\\
                a' \in \{0,[\bar{A}_{ih},\infty)\}
                }
            }
        u\left(c\right)+  \beta \int v_{ih}\left(a',z'\right)f(z'|z)dz'\\
    \end{split}
\end{align}

Using Array, fonts are larger under max:
\begin{align}
    \begin{split}
        \label{eq:Value}
        v_{ih}\left(a,z\right)
        =
        \max_{
            \begin{array}{cc}
                c>0\\
                a' \in \{0,[\bar{A}_{ih},\infty)\}\\
            \end{array}
            }
        u\left(c\right)+  \beta \int v_{ih}\left(a',z'\right)f(z'|z)dz'\\
    \end{split}
\end{align}

\pagebreak

\section{Multiple Lines}

\begin{verbatim}
\begin{align}
    x = y + z +
    \left\{
      \begin{array}{l}
            a \\
            + b \\
            + c \\
            + d \\
      \end{array}
    \right\}
\end{align}
\end{verbatim}

\begin{align}
    x = y + z +
    \left\{
      \begin{array}{l}
            a \\
            + b \\
            + c \\
            + d \\
      \end{array}
    \right\}
\end{align}

\end{document}
