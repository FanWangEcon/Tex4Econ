\documentclass[12pt,english]{article}
\usepackage[utf8]{inputenc}
\usepackage[T1]{fontenc}
\usepackage{mathtools}

% url package
\usepackage[colorlinks=true,
            linkcolor=blue,
            urlcolor=blue,
            anchorcolor = blue,
            citecolor=gray]{hyperref}

% Titling and Author
\title{Latex Example, Equation Cases}
\author{\href{https://fanwangecon.github.io/}{Fan Wang}\thanks{https://fanwangecon.github.io, repository: \href{https://fanwangecon.github.io/Tex4Econ/}{Tex4Econ}}}
\date{\today}
\begin{document}

\maketitle

\section{Two Cases}

\begin{verbatim}
  x =
  \begin{cases*}
      -x           & if  $x < 0 $  \\
      \phantom{-}x & if  $x\ge 0$
  \end{cases*}
\end{verbatim}

$$
x =
\begin{cases*}
    -x           & if  $x < 0 $  \\
    \phantom{-}x & if  $x\ge 0$
\end{cases*}
$$

\begin{align}
x =
\begin{cases*}
-x           & if  $x < 0 $  \\
\phantom{-}x & if  $x\ge 0$
\end{cases*}
\end{align}

\pagebreak

\section{Two Cases, Same Line}

\begin{verbatim}
  \begin{equation*}
  f(x) = \begin{cases}
               0  & \text{if } x < 0 \\
               1  & \text{if } x \ge 0
         \end{cases} \quad
  g(x) = \begin{cases}
               f(x)+1  & \text{if } x < 0 \\
               f(x)-1  & \text{if } x \ge 0
         \end{cases}
  \end{equation*}
\end{verbatim}

\begin{equation*}
f(x) = \begin{cases}
             0  & \text{if } x < 0 \\
             1  & \text{if } x \ge 0
       \end{cases} \quad
g(x) = \begin{cases}
             f(x)+1  & \text{if } x < 0 \\
             f(x)-1  & \text{if } x \ge 0
       \end{cases}
\end{equation*}

\begin{align}
f(x) = \begin{cases}
             0  & \text{if } x < 0 \\
             1  & \text{if } x \ge 0
       \end{cases} \quad
g(x) = \begin{cases}
             f(x)+1  & \text{if } x < 0 \\
             f(x)-1  & \text{if } x \ge 0
       \end{cases}
\end{align}

case star
\begin{align}
f(x) =
\begin{cases*}
0  & if $x < 0$ \\
1  & if $x \ge 0$
\end{cases*}
\quad
g(x) =
\begin{cases*}
f(x)+1  & if $ x < 0$ \\
f(x)-1  & if $ x \ge 0$
\end{cases*}
\end{align}

\pagebreak

\pagebreak

\section{Cases with Fraction Large Using Array dcases}

Here, we compare the difference between using dcases and cases with fractions.

\subsection{cases}

\begin{align}
f(x) = \begin{cases*}
             \frac{a+b}{c+d}  & \text{if } x < 0 \\
             1  & \text{if } x \ge 0
       \end{cases*}
\end{align}

\subsection{dcases}
Fraction show up larger

\begin{align}
f(x) = \begin{dcases*}
             \frac{a+b}{c+d}  & \text{if } x < 0 \\
             1  & \text{if } x \ge 0
       \end{dcases*}
\end{align}


\end{document}
