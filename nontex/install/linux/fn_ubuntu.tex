\PassOptionsToPackage{unicode=true}{hyperref} % options for packages loaded elsewhere
\PassOptionsToPackage{hyphens}{url}
\PassOptionsToPackage{dvipsnames,svgnames*,x11names*}{xcolor}
%
\documentclass[]{article}
\usepackage{lmodern}
\usepackage{amssymb,amsmath}
\usepackage{ifxetex,ifluatex}
\usepackage{fixltx2e} % provides \textsubscript
\ifnum 0\ifxetex 1\fi\ifluatex 1\fi=0 % if pdftex
  \usepackage[T1]{fontenc}
  \usepackage[utf8]{inputenc}
  \usepackage{textcomp} % provides euro and other symbols
\else % if luatex or xelatex
  \usepackage{unicode-math}
  \defaultfontfeatures{Ligatures=TeX,Scale=MatchLowercase}
\fi
% use upquote if available, for straight quotes in verbatim environments
\IfFileExists{upquote.sty}{\usepackage{upquote}}{}
% use microtype if available
\IfFileExists{microtype.sty}{%
\usepackage[]{microtype}
\UseMicrotypeSet[protrusion]{basicmath} % disable protrusion for tt fonts
}{}
\IfFileExists{parskip.sty}{%
\usepackage{parskip}
}{% else
\setlength{\parindent}{0pt}
\setlength{\parskip}{6pt plus 2pt minus 1pt}
}
\usepackage{xcolor}
\usepackage{hyperref}
\hypersetup{
            pdftitle={Example and Tutorial for Atom, Git, Github, etc Set-up for Linux},
            colorlinks=true,
            linkcolor=Maroon,
            citecolor=Blue,
            urlcolor=blue,
            breaklinks=true}
\urlstyle{same}  % don't use monospace font for urls
\usepackage[margin=1in]{geometry}
\usepackage{color}
\usepackage{fancyvrb}
\newcommand{\VerbBar}{|}
\newcommand{\VERB}{\Verb[commandchars=\\\{\}]}
\DefineVerbatimEnvironment{Highlighting}{Verbatim}{commandchars=\\\{\}}
% Add ',fontsize=\small' for more characters per line
\usepackage{framed}
\definecolor{shadecolor}{RGB}{248,248,248}
\newenvironment{Shaded}{\begin{snugshade}}{\end{snugshade}}
\newcommand{\AlertTok}[1]{\textcolor[rgb]{0.94,0.16,0.16}{#1}}
\newcommand{\AnnotationTok}[1]{\textcolor[rgb]{0.56,0.35,0.01}{\textbf{\textit{#1}}}}
\newcommand{\AttributeTok}[1]{\textcolor[rgb]{0.77,0.63,0.00}{#1}}
\newcommand{\BaseNTok}[1]{\textcolor[rgb]{0.00,0.00,0.81}{#1}}
\newcommand{\BuiltInTok}[1]{#1}
\newcommand{\CharTok}[1]{\textcolor[rgb]{0.31,0.60,0.02}{#1}}
\newcommand{\CommentTok}[1]{\textcolor[rgb]{0.56,0.35,0.01}{\textit{#1}}}
\newcommand{\CommentVarTok}[1]{\textcolor[rgb]{0.56,0.35,0.01}{\textbf{\textit{#1}}}}
\newcommand{\ConstantTok}[1]{\textcolor[rgb]{0.00,0.00,0.00}{#1}}
\newcommand{\ControlFlowTok}[1]{\textcolor[rgb]{0.13,0.29,0.53}{\textbf{#1}}}
\newcommand{\DataTypeTok}[1]{\textcolor[rgb]{0.13,0.29,0.53}{#1}}
\newcommand{\DecValTok}[1]{\textcolor[rgb]{0.00,0.00,0.81}{#1}}
\newcommand{\DocumentationTok}[1]{\textcolor[rgb]{0.56,0.35,0.01}{\textbf{\textit{#1}}}}
\newcommand{\ErrorTok}[1]{\textcolor[rgb]{0.64,0.00,0.00}{\textbf{#1}}}
\newcommand{\ExtensionTok}[1]{#1}
\newcommand{\FloatTok}[1]{\textcolor[rgb]{0.00,0.00,0.81}{#1}}
\newcommand{\FunctionTok}[1]{\textcolor[rgb]{0.00,0.00,0.00}{#1}}
\newcommand{\ImportTok}[1]{#1}
\newcommand{\InformationTok}[1]{\textcolor[rgb]{0.56,0.35,0.01}{\textbf{\textit{#1}}}}
\newcommand{\KeywordTok}[1]{\textcolor[rgb]{0.13,0.29,0.53}{\textbf{#1}}}
\newcommand{\NormalTok}[1]{#1}
\newcommand{\OperatorTok}[1]{\textcolor[rgb]{0.81,0.36,0.00}{\textbf{#1}}}
\newcommand{\OtherTok}[1]{\textcolor[rgb]{0.56,0.35,0.01}{#1}}
\newcommand{\PreprocessorTok}[1]{\textcolor[rgb]{0.56,0.35,0.01}{\textit{#1}}}
\newcommand{\RegionMarkerTok}[1]{#1}
\newcommand{\SpecialCharTok}[1]{\textcolor[rgb]{0.00,0.00,0.00}{#1}}
\newcommand{\SpecialStringTok}[1]{\textcolor[rgb]{0.31,0.60,0.02}{#1}}
\newcommand{\StringTok}[1]{\textcolor[rgb]{0.31,0.60,0.02}{#1}}
\newcommand{\VariableTok}[1]{\textcolor[rgb]{0.00,0.00,0.00}{#1}}
\newcommand{\VerbatimStringTok}[1]{\textcolor[rgb]{0.31,0.60,0.02}{#1}}
\newcommand{\WarningTok}[1]{\textcolor[rgb]{0.56,0.35,0.01}{\textbf{\textit{#1}}}}
\usepackage{graphicx,grffile}
\makeatletter
\def\maxwidth{\ifdim\Gin@nat@width>\linewidth\linewidth\else\Gin@nat@width\fi}
\def\maxheight{\ifdim\Gin@nat@height>\textheight\textheight\else\Gin@nat@height\fi}
\makeatother
% Scale images if necessary, so that they will not overflow the page
% margins by default, and it is still possible to overwrite the defaults
% using explicit options in \includegraphics[width, height, ...]{}
\setkeys{Gin}{width=\maxwidth,height=\maxheight,keepaspectratio}
\setlength{\emergencystretch}{3em}  % prevent overfull lines
\providecommand{\tightlist}{%
  \setlength{\itemsep}{0pt}\setlength{\parskip}{0pt}}
\setcounter{secnumdepth}{5}
% Redefines (sub)paragraphs to behave more like sections
\ifx\paragraph\undefined\else
\let\oldparagraph\paragraph
\renewcommand{\paragraph}[1]{\oldparagraph{#1}\mbox{}}
\fi
\ifx\subparagraph\undefined\else
\let\oldsubparagraph\subparagraph
\renewcommand{\subparagraph}[1]{\oldsubparagraph{#1}\mbox{}}
\fi

% set default figure placement to htbp
\makeatletter
\def\fps@figure{htbp}
\makeatother


\title{Example and Tutorial for Atom, Git, Github, etc Set-up for Linux}
\author{}
\date{\vspace{-2.5em}}

\begin{document}
\maketitle

Go back to \href{http://fanwangecon.github.io/}{fan}'s
\href{http://fanwangecon.github.io/Tex4Econ/}{Tex4Econ and Miscellaneous
Repository}.

\hypertarget{atom-install}{%
\section{Atom Install}\label{atom-install}}

Can be directly installed from Ubuntu store.

\begin{Shaded}
\begin{Highlighting}[]
\CommentTok{# install atom}
\FunctionTok{sudo}\NormalTok{ apt-get install atom}
\CommentTok{# install various atom packages that are necessary}
\FunctionTok{apm}\NormalTok{ install vim-mode-plus}
\end{Highlighting}
\end{Shaded}

\begin{itemize}
\item
  \emph{personal access token}: d8a293d30ef620a8b27ab0ad6294cdd8942a0b6f
\item
  \emph{Gist id}: 754fa9ca57eb7ddc1ee89775f934b59f
\item
  \emph{personal access token}: b36b4d26ff2b5b626a1d7f66e3bc60c582a85c21
\item
  \emph{Gist id}: a883a7ba62661dbe9b907d24b651b780
\end{itemize}

\hypertarget{git-install}{%
\section{Git Install}\label{git-install}}

\begin{enumerate}
\def\labelenumi{\arabic{enumi}.}
\tightlist
\item
  install git
\item
  once rsa set up, sync repo
\end{enumerate}

\begin{Shaded}
\begin{Highlighting}[]
\CommentTok{# install git}
\FunctionTok{sudo}\NormalTok{ apt-get install git}
\CommentTok{# generate repo directory and sync}
\FunctionTok{mkdir}\NormalTok{ ~/PyFan}
\CommentTok{# mkdir fanwangecon.github.io PyFan Tex4Econ R4Econ M4Econ Py4Econ Teaching}
\BuiltInTok{cd}\NormalTok{ ~/PyFan}
\FunctionTok{git}\NormalTok{ init}
\CommentTok{# git config only needs to happen once, info stored under ~/.gitconfig}
\FunctionTok{git}\NormalTok{ config --global user.name }\StringTok{"Fan Wang"}
\FunctionTok{git}\NormalTok{ config --global user.email wangfanbsg75@live.com}
\FunctionTok{git}\NormalTok{ remote add github git@github.com:FanWangEcon/PyFan.git}
\FunctionTok{git}\NormalTok{ pull github master}
\end{Highlighting}
\end{Shaded}

\hypertarget{github-and-local-repo}{%
\section{Github and local repo}\label{github-and-local-repo}}

\begin{enumerate}
\def\labelenumi{\arabic{enumi}.}
\tightlist
\item
  generate rsa

  \begin{itemize}
  \tightlist
  \item
    ssh-keygen -t rsa -C
    ``\href{mailto:wangfanbsg75@live.com}{\nolinkurl{wangfanbsg75@live.com}}'',
    when prompted, do not enter ``file in which save the key'', when
    prompted for passphrase, enter ``WHATEVERPASSWORDIS''
  \end{itemize}
\item
  copy key
\item
  log on to github ssh section, generate new ssh rsa key
\end{enumerate}

\begin{Shaded}
\begin{Highlighting}[]
\FunctionTok{ssh-keygen}\NormalTok{ -t rsa -C }\StringTok{"wangfanbsg75@live.com"}
\FunctionTok{cat}\NormalTok{ ~/.ssh/id_rsa.pub}
\end{Highlighting}
\end{Shaded}

\hypertarget{conda-install}{%
\section{Conda Install}\label{conda-install}}

conda install linux

\begin{enumerate}
\def\labelenumi{\arabic{enumi}.}
\tightlist
\item
  wget to download from url to download folder
\item
  bash to install
\item
  follow instructions, type yes
\item
  source \textasciitilde{}/.bashrc
\end{enumerate}

\begin{Shaded}
\begin{Highlighting}[]
\CommentTok{# To remove conda Fully}
\FunctionTok{rm}\NormalTok{ -rf ~/anaconda3}

\CommentTok{# could be saved in current folder: pwd}
\CommentTok{# could be saved in download folder: ~/Downloads}
\FunctionTok{wget}\NormalTok{ https://repo.anaconda.com/archive/Anaconda3-2019.10-Linux-x86_64.sh}
\CommentTok{# install file}
\FunctionTok{bash} \StringTok{"Anaconda3-2019.10-Linux-x86_64.sh"}
\CommentTok{# refresh}
\BuiltInTok{source}\NormalTok{ ~/.bashrc}

\CommentTok{# show all installed packages under current envir}
\ExtensionTok{conda}\NormalTok{ list}
\CommentTok{# see where key files are installed}
\FunctionTok{which}\NormalTok{ python}
\FunctionTok{which}\NormalTok{ conda}
\CommentTok{# use conda-forge as main channel and update all}
\ExtensionTok{/*}\NormalTok{ conda config --add channels conda-forge */}
\ExtensionTok{conda}\NormalTok{ update --all}
\end{Highlighting}
\end{Shaded}

Additional statistics and related models to install:

\begin{Shaded}
\begin{Highlighting}[]
\NormalTok{conda install }\OperatorTok{-}\NormalTok{y statsmodels datashape seaborn}
\CommentTok{# conda install statsmodels}
\CommentTok{# conda install datashape}
\CommentTok{# conda install seaborn}

\NormalTok{conda install }\OperatorTok{-}\NormalTok{c conda}\OperatorTok{-}\NormalTok{forge }\OperatorTok{-}\NormalTok{y interpolation awscli}
\CommentTok{# conda install -c conda-forge interpolation}
\CommentTok{# conda install -c conda-forge awscli}

\NormalTok{conda install }\OperatorTok{-}\NormalTok{c anaconda }\OperatorTok{-}\NormalTok{y boto3}

\CommentTok{# conda install -c r r-irkernel}
\end{Highlighting}
\end{Shaded}

\hypertarget{install-pycharm}{%
\subsection{Install PyCharm}\label{install-pycharm}}

PyCharm can be installed from Ubuntu apps.

\begin{Shaded}
\begin{Highlighting}[]
\OperatorTok{<}\NormalTok{!}\ExtensionTok{--}\NormalTok{ for Debian --}\OperatorTok{>}
\FunctionTok{sudo}\NormalTok{ tar xfz pycharm-*.tar.gz -C /opt/}
\BuiltInTok{cd}\NormalTok{ /opt/pycharm-*/bin}
\ExtensionTok{./pycharm.sh}
\end{Highlighting}
\end{Shaded}

\hypertarget{r-install}{%
\section{R install}\label{r-install}}

R could be installed first as below. Or follow the
\href{https://fanwangecon.github.io/Tex4Econ/nontex/install/windows/fn_installations.html}{instructions
on this page} to install from conda an environment for R, with
associated R-studio.

\begin{Shaded}
\begin{Highlighting}[]
\CommentTok{# Debian R is maintained by Johannes Ranke, copied from https://cran.r-project.org/bin/linux/debian/:}
\ExtensionTok{apt-key}\NormalTok{ adv --keyserver keys.gnupg.net --recv-key }\StringTok{'E19F5F87128899B192B1A2C2AD5F960A256A04AF'}
\CommentTok{# Add to source.list, for debian stretch (9)}
\CommentTok{# sudo su added for security issue as super-user}
\FunctionTok{sudo}\NormalTok{ su -c }\StringTok{"sudo echo 'deb http://cloud.r-project.org/bin/linux/debian stretch-cran35/' >> /etc/apt/sources.list"}
\CommentTok{# if added wrong lines, delete 3rd line}
\FunctionTok{sudo}\NormalTok{ sed }\StringTok{'3d'}\NormalTok{ /etc/apt/sources.list}

\CommentTok{# Update and Install R, should say updated from cloud.r}
\FunctionTok{sudo}\NormalTok{ apt-get update}
\FunctionTok{sudo}\NormalTok{ apt-get install r-base r-base-dev}
\end{Highlighting}
\end{Shaded}

\hypertarget{install-other-programs-and-packages}{%
\section{Install Other Programs and
Packages}\label{install-other-programs-and-packages}}

Other key programs and packages to install.

\begin{itemize}
\tightlist
\item
  vim: faster editor, on lower resource machines, atom is slow and
  typing could feel laggy.
\item
  htop: for resource monitoring
\end{itemize}

\begin{verbatim}
sudo apt-get install vim
sudo apt-get install sublime-text
sudo apt-get install htop
\end{verbatim}

\end{document}
