\PassOptionsToPackage{unicode=true}{hyperref} % options for packages loaded elsewhere
\PassOptionsToPackage{hyphens}{url}
\PassOptionsToPackage{dvipsnames,svgnames*,x11names*}{xcolor}
%
\documentclass[]{article}
\usepackage{lmodern}
\usepackage{amssymb,amsmath}
\usepackage{ifxetex,ifluatex}
\usepackage{fixltx2e} % provides \textsubscript
\ifnum 0\ifxetex 1\fi\ifluatex 1\fi=0 % if pdftex
  \usepackage[T1]{fontenc}
  \usepackage[utf8]{inputenc}
  \usepackage{textcomp} % provides euro and other symbols
\else % if luatex or xelatex
  \usepackage{unicode-math}
  \defaultfontfeatures{Ligatures=TeX,Scale=MatchLowercase}
\fi
% use upquote if available, for straight quotes in verbatim environments
\IfFileExists{upquote.sty}{\usepackage{upquote}}{}
% use microtype if available
\IfFileExists{microtype.sty}{%
\usepackage[]{microtype}
\UseMicrotypeSet[protrusion]{basicmath} % disable protrusion for tt fonts
}{}
\IfFileExists{parskip.sty}{%
\usepackage{parskip}
}{% else
\setlength{\parindent}{0pt}
\setlength{\parskip}{6pt plus 2pt minus 1pt}
}
\usepackage{xcolor}
\usepackage{hyperref}
\hypersetup{
            pdftitle={Example and Tutorial for Vim Editor (Linux)},
            colorlinks=true,
            linkcolor=Maroon,
            citecolor=Blue,
            urlcolor=blue,
            breaklinks=true}
\urlstyle{same}  % don't use monospace font for urls
\usepackage[margin=1in]{geometry}
\usepackage{color}
\usepackage{fancyvrb}
\newcommand{\VerbBar}{|}
\newcommand{\VERB}{\Verb[commandchars=\\\{\}]}
\DefineVerbatimEnvironment{Highlighting}{Verbatim}{commandchars=\\\{\}}
% Add ',fontsize=\small' for more characters per line
\usepackage{framed}
\definecolor{shadecolor}{RGB}{248,248,248}
\newenvironment{Shaded}{\begin{snugshade}}{\end{snugshade}}
\newcommand{\AlertTok}[1]{\textcolor[rgb]{0.94,0.16,0.16}{#1}}
\newcommand{\AnnotationTok}[1]{\textcolor[rgb]{0.56,0.35,0.01}{\textbf{\textit{#1}}}}
\newcommand{\AttributeTok}[1]{\textcolor[rgb]{0.77,0.63,0.00}{#1}}
\newcommand{\BaseNTok}[1]{\textcolor[rgb]{0.00,0.00,0.81}{#1}}
\newcommand{\BuiltInTok}[1]{#1}
\newcommand{\CharTok}[1]{\textcolor[rgb]{0.31,0.60,0.02}{#1}}
\newcommand{\CommentTok}[1]{\textcolor[rgb]{0.56,0.35,0.01}{\textit{#1}}}
\newcommand{\CommentVarTok}[1]{\textcolor[rgb]{0.56,0.35,0.01}{\textbf{\textit{#1}}}}
\newcommand{\ConstantTok}[1]{\textcolor[rgb]{0.00,0.00,0.00}{#1}}
\newcommand{\ControlFlowTok}[1]{\textcolor[rgb]{0.13,0.29,0.53}{\textbf{#1}}}
\newcommand{\DataTypeTok}[1]{\textcolor[rgb]{0.13,0.29,0.53}{#1}}
\newcommand{\DecValTok}[1]{\textcolor[rgb]{0.00,0.00,0.81}{#1}}
\newcommand{\DocumentationTok}[1]{\textcolor[rgb]{0.56,0.35,0.01}{\textbf{\textit{#1}}}}
\newcommand{\ErrorTok}[1]{\textcolor[rgb]{0.64,0.00,0.00}{\textbf{#1}}}
\newcommand{\ExtensionTok}[1]{#1}
\newcommand{\FloatTok}[1]{\textcolor[rgb]{0.00,0.00,0.81}{#1}}
\newcommand{\FunctionTok}[1]{\textcolor[rgb]{0.00,0.00,0.00}{#1}}
\newcommand{\ImportTok}[1]{#1}
\newcommand{\InformationTok}[1]{\textcolor[rgb]{0.56,0.35,0.01}{\textbf{\textit{#1}}}}
\newcommand{\KeywordTok}[1]{\textcolor[rgb]{0.13,0.29,0.53}{\textbf{#1}}}
\newcommand{\NormalTok}[1]{#1}
\newcommand{\OperatorTok}[1]{\textcolor[rgb]{0.81,0.36,0.00}{\textbf{#1}}}
\newcommand{\OtherTok}[1]{\textcolor[rgb]{0.56,0.35,0.01}{#1}}
\newcommand{\PreprocessorTok}[1]{\textcolor[rgb]{0.56,0.35,0.01}{\textit{#1}}}
\newcommand{\RegionMarkerTok}[1]{#1}
\newcommand{\SpecialCharTok}[1]{\textcolor[rgb]{0.00,0.00,0.00}{#1}}
\newcommand{\SpecialStringTok}[1]{\textcolor[rgb]{0.31,0.60,0.02}{#1}}
\newcommand{\StringTok}[1]{\textcolor[rgb]{0.31,0.60,0.02}{#1}}
\newcommand{\VariableTok}[1]{\textcolor[rgb]{0.00,0.00,0.00}{#1}}
\newcommand{\VerbatimStringTok}[1]{\textcolor[rgb]{0.31,0.60,0.02}{#1}}
\newcommand{\WarningTok}[1]{\textcolor[rgb]{0.56,0.35,0.01}{\textbf{\textit{#1}}}}
\usepackage{graphicx,grffile}
\makeatletter
\def\maxwidth{\ifdim\Gin@nat@width>\linewidth\linewidth\else\Gin@nat@width\fi}
\def\maxheight{\ifdim\Gin@nat@height>\textheight\textheight\else\Gin@nat@height\fi}
\makeatother
% Scale images if necessary, so that they will not overflow the page
% margins by default, and it is still possible to overwrite the defaults
% using explicit options in \includegraphics[width, height, ...]{}
\setkeys{Gin}{width=\maxwidth,height=\maxheight,keepaspectratio}
\setlength{\emergencystretch}{3em}  % prevent overfull lines
\providecommand{\tightlist}{%
  \setlength{\itemsep}{0pt}\setlength{\parskip}{0pt}}
\setcounter{secnumdepth}{5}
% Redefines (sub)paragraphs to behave more like sections
\ifx\paragraph\undefined\else
\let\oldparagraph\paragraph
\renewcommand{\paragraph}[1]{\oldparagraph{#1}\mbox{}}
\fi
\ifx\subparagraph\undefined\else
\let\oldsubparagraph\subparagraph
\renewcommand{\subparagraph}[1]{\oldsubparagraph{#1}\mbox{}}
\fi

% set default figure placement to htbp
\makeatletter
\def\fps@figure{htbp}
\makeatother


\title{Example and Tutorial for Vim Editor (Linux)}
\author{}
\date{\vspace{-2.5em}}

\begin{document}
\maketitle

Go back to \href{http://fanwangecon.github.io/}{fan}'s
\href{http://fanwangecon.github.io/Tex4Econ/}{Tex4Econ and Miscellaneous
Repository}.

\hypertarget{vim-installations}{%
\section{Vim Installations}\label{vim-installations}}

Vim ships with Debian. Also install neo-vim:

\begin{Shaded}
\begin{Highlighting}[]
\ExtensionTok{vim}
\FunctionTok{sudo}\NormalTok{ apt-get install neovim}
\end{Highlighting}
\end{Shaded}

Just installing onedark, neovim seems to work better than vim. Vim has
delete lag, and extra white space rows at the bottom of the screen that
could take up very large screen segment. Neovim with onedark does not
have this issue, could be due to some basic setting differences.

\hypertarget{install-vim-plug-plug-ins-manager-for-vim}{%
\subsection{Install Vim-Plug Plug-ins manager for
Vim}\label{install-vim-plug-plug-ins-manager-for-vim}}

Follow the
\href{https://github.com/junegunn/vim-plug/wiki/tutorial}{instructions
here} for installations. Several steps:

\begin{enumerate}
\def\labelenumi{\arabic{enumi}.}
\tightlist
\item
  generate \textasciitilde{}/.vim and \textasciitilde{}/.vimrc folders
\item
  curl plug.vim to \textasciitilde{}/.vim/autoload
\item
  update \textasciitilde{}/.vimrc file with instructions regarding which
  files to install.
\item
  open vim, type: \emph{:PlugInstall} or \emph{:PlugUpdate}, should
  start seeing updating packages.
\item
  type: \emph{:scriptnames} to see loaded plugins
\item
  just comment out or delete plug lines for packages no longer needed.
\item
  how to install a particular package? See
  \href{https://vimawesome.com/plugin/onedark-vim}{vimawesome}
\end{enumerate}

\begin{Shaded}
\begin{Highlighting}[]
\CommentTok{# curl plug.vim}
\ExtensionTok{curl}\NormalTok{ -fLo ~/.vim/autoload/plug.vim --create-dirs \textbackslash{}}
\NormalTok{    https://raw.githubusercontent.com/junegunn/vim-plug/master/plug.vim}

\CommentTok{# create vimrc file}
\BuiltInTok{cd} \VariableTok{$HOME}
\ExtensionTok{vim}\NormalTok{ .vimrc}

\CommentTok{# paste into file the text below.}
\end{Highlighting}
\end{Shaded}

Example .vimrc file with only one package one dark atom visual for vim:

\begin{Shaded}
\begin{Highlighting}[]
\KeywordTok{if} \ExtensionTok{empty}\NormalTok{(glob(}\StringTok{'~/.vim/autoload/plug.vim'}\NormalTok{))}
        \ExtensionTok{silent}\NormalTok{ !curl -fLo ~/.vim/autoload/plug.vim --create-dirs}
                \ExtensionTok{\textbackslash{} https}\NormalTok{://raw.githubusercontent.com/junegunn/vim-plug/master/plug.vim}
        \ExtensionTok{autocmd}\NormalTok{ VimEnter * PlugInstall --sync }\KeywordTok{|} \BuiltInTok{source} \VariableTok{$MYVIMRC}
\ExtensionTok{endif}

\StringTok{" Plugins will be downloaded under the specified directory.}
\StringTok{call plug#begin('~/.vim/plugged')}

\StringTok{"} \ExtensionTok{Declare}\NormalTok{ the list of plugins.}
\ExtensionTok{Plug} \StringTok{'joshdick/onedark.vim'}

\StringTok{" List ends here. Plugins become visible to Vim after this call.}
\StringTok{call plug#end()}

\StringTok{"""""""""""""""""""""""""""""""""""""""""""""""""}
\StringTok{" Settings}
\StringTok{"""""""""""""""""""""""""""""""""""""""""""""""""}
\ExtensionTok{syntax}\NormalTok{ on}
\ExtensionTok{colorscheme}\NormalTok{ onedark}
\end{Highlighting}
\end{Shaded}

\hypertarget{install-vim-plug-plug-ins-manager-for-neovim}{%
\subsection{Install Vim-Plug Plug-ins manager for
Neovim}\label{install-vim-plug-plug-ins-manager-for-neovim}}

Follow the
\href{https://github.com/junegunn/vim-plug/wiki/tutorial}{instructions
here} for installations. Very similar steps as above for Vim:

\begin{enumerate}
\def\labelenumi{\arabic{enumi}.}
\tightlist
\item
  curl \emph{plug.vim} to \emph{.local/share/nvim}. Directory different
  than vim
\item
  create nvim folder under .config, vimrc contents should be in this
  file: \emph{\textasciitilde{}/.config/nvim/init.vim}
\item
  enter nvim and \emph{:PlugInstall}
\end{enumerate}

\begin{Shaded}
\begin{Highlighting}[]
\CommentTok{# Neovim (~/.local/share/nvim/site/autoload)}
\ExtensionTok{curl}\NormalTok{ -fLo ~/.local/share/nvim/site/autoload/plug.vim --create-dirs \textbackslash{}}
\NormalTok{    https://raw.githubusercontent.com/junegunn/vim-plug/master/plug.vim}

\CommentTok{# create vimrc file}
\FunctionTok{mkdir}\NormalTok{ ~/.config/nvim}
\ExtensionTok{nvim}\NormalTok{ ~/.config/nvim/init.vim}

\CommentTok{# paste into file the text below.}
\end{Highlighting}
\end{Shaded}

Vimrc file with atom styling:

\begin{verbatim}
" Plugins will be downloaded under the specified directory.
call plug#begin('~/.local/share/nvim/site/plugged')

" Declare the list of plugins.
Plug 'vim-airline/vim-airline'

" Visual Settings
Plug 'junegunn/goyo.vim'
Plug 'junegunn/limelight.vim'

" Color theme plugins
Plug 'joshdick/onedark.vim'

" List ends here. Plugins become visible to Vim after this call.
call plug#end()

"""""""""""""""""""""""""""""""""""""""""""""""""
" Color Settings
"""""""""""""""""""""""""""""""""""""""""""""""""
syntax on

colorscheme onedark

"""""""""""""""""""""""""""""""""""""""""""""""""
" Visual Settings
"""""""""""""""""""""""""""""""""""""""""""""""""
" For Goyo
let g:goyo_width=85

" For LimeLight
" Color name (:help cterm-colors) or ANSI code
let g:limelight_conceal_ctermfg = 'gray'
let g:limelight_conceal_ctermfg = 240
" Color name (:help gui-colors) or RGB color
let g:limelight_conceal_guifg = 'DarkGray'
let g:limelight_conceal_guifg = '#777777'
" highlight line
let g:limelight_bop = '^.*$'
" let g:limelight_eop = '\n'
let g:limelight_paragraph_span = 0

"""""""""""""""""""""""""""""""""""""""""""""""""
" UI Settings
"""""""""""""""""""""""""""""""""""""""""""""""""
" Keep cursor in the middle of the page, useful for editing text
set so=999
" Turn limelight on by default
" autocmd VimEnter * Limelight
" Turn Goyo on by default
autocmd VimEnter * Goyo
" autocmd VimEnter * AirlineToggle
" In Goyo, if airline is turned on, do nto show scratch area
" autocmd! User GoyoEnter nested set eventignore=FocusGained
" autocmd! User GoyoLeave nested set eventignore=
\end{verbatim}

\hypertarget{install-neovim-in-windows}{%
\subsection{Install Neovim in Windows}\label{install-neovim-in-windows}}

Install and access neovim all from windows powershell. Type in
\emph{powershell} in regular command prompt for example.

\hypertarget{install-neovim-and-vim-plug-in-windows}{%
\subsubsection{Install Neovim and Vim-Plug in
Windows}\label{install-neovim-and-vim-plug-in-windows}}

\begin{verbatim}
# install scoop and use scoop
scoop install neovim

# follow vim-plug instructions
md ~\AppData\Local\nvim\autoload
$uri = 'https://raw.githubusercontent.com/junegunn/vim-plug/master/plug.vim'
(New-Object Net.WebClient).DownloadFile(
  $uri,
  $ExecutionContext.SessionState.Path.GetUnresolvedProviderPathFromPSPath(
    "~\AppData\Local\nvim\autoload\plug.vim"
  )
)
\end{verbatim}

\hypertarget{set-up-init.vim-file-in-windows}{%
\subsubsection{Set up init.vim File in
Windows}\label{set-up-init.vim-file-in-windows}}

Inside powershell, create init.vim file:

\begin{verbatim}
cd ~\AppData\Local\nvim\
nvim init.vim
\end{verbatim}

paste these in \emph{:set paste}:

\begin{verbatim}
" Plugins will be downloaded under the specified directory.
call plug#begin('~/AppData/Local/nvim/site/plugged')

" Declare the list of plugins.
Plug 'vim-airline/vim-airline'

" Markdown
Plug 'godlygeek/tabular'
Plug 'plasticboy/vim-markdown'

" Visual Settings
Plug 'junegunn/goyo.vim'
Plug 'junegunn/limelight.vim'
Plug 'reedes/vim-pencil'

" Color theme plugins
Plug 'joshdick/onedark.vim'
Plug 'morhetz/gruvbox'
Plug 'junegunn/seoul256.vim'
Plug 'jaredgorski/spacecamp'
Plug 'lifepillar/vim-solarized8'
Plug 'reedes/vim-colors-pencil'

" List ends here. Plugins become visible to Vim after this call.
call plug#end()

"""""""""""""""""""""""""""""""""""""""""""""""""
" Color Settings
"""""""""""""""""""""""""""""""""""""""""""""""""
syntax on

" onedark:
colorscheme onedark

" seoul256:
" seoul256 (dark), 236 is darkest
" let g:seoul256_background = 236
" colorscheme seoul256

" spacecamp:
" colorscheme spacecamp

" vim-solarized8:
" set background=dark
" colorscheme solarized8

" vim-colors-pencil:
" let g:pencil_higher_contrast_ui = 0
" colorscheme pencil
" set background=dark

"""""""""""""""""""""""""""""""""""""""""""""""""
" Visual Settings
"""""""""""""""""""""""""""""""""""""""""""""""""
" For Goyo
let g:goyo_width=85

" For LimeLight
" Color name (:help cterm-colors) or ANSI code
let g:limelight_conceal_ctermfg = 'gray'
let g:limelight_conceal_ctermfg = 240
" Color name (:help gui-colors) or RGB color
let g:limelight_conceal_guifg = 'DarkGray'
let g:limelight_conceal_guifg = '#777777'
" if Limelight should highlight line rather than paragraph
let g:limelight_bop = '^.*$'
" let g:limelight_eop = '\n'
let g:limelight_paragraph_span = 0

"""""""""""""""""""""""""""""""""""""""""""""""""
" UI Settings
"""""""""""""""""""""""""""""""""""""""""""""""""
" Keep cursor in the middle of the page, useful for editing text
set so=999
" Turn limelight on by default
" autocmd VimEnter * Limelight
" Turn Goyo on by default
autocmd VimEnter * Goyo

"""""""""""""""""""""""""""""""""""""""""""""""""
" Markdown **, * etc settings
"""""""""""""""""""""""""""""""""""""""""""""""""
" set concealcursor=n
" set conceallevel=3
" hi Asterisks NONE
" hi AsteriskBold  cterm=bold gui=bold
" syn match Asterisks contained "**" conceal
" syn match AsteriskBold "\\\@<!\*\*[^"*|]\+\*\*" contains=Asterisks
" autocmd VimEnter * highlight Normal ctermbg=black
" ctermfg=grey

"""""""""""""""""""""""""""""""""""""""""""""""""
" Other Settings
"""""""""""""""""""""""""""""""""""""""""""""""""
set paste               " to enable pasting in windows
\end{verbatim}

\begin{verbatim}
nvim "C:/Users/d674a/Tex4Econ/_other/equation/multilines.tex"
\end{verbatim}

\hypertarget{full-screen-vim-window-in-windows}{%
\subsubsection{Full Screen Vim Window in
Windows}\label{full-screen-vim-window-in-windows}}

It is nice to have full screen, no distraction work experience. To set
command prompt to full screen (to get rid of scrolling bar on the right
in full screen), use windows command prompt which allows for full
screen, right click on window top border, choose property, then choose
layout, then set height and width of window size and buffer size to be
the same. This will allow for full screen experience.

\hypertarget{vim-basics}{%
\section{Vim Basics}\label{vim-basics}}

\begin{Shaded}
\begin{Highlighting}[]
\CommentTok{# to open vim}
\ExtensionTok{vim}\NormalTok{ -v}
\CommentTok{# to exit vim}
\NormalTok{:}\ExtensionTok{q}\NormalTok{!}
\CommentTok{# open up file from google drive}
\ExtensionTok{vim} \StringTok{"/mnt/chromeos/GoogleDrive/MyDrive/Reading Papers/Reading/Referee/pa_2019_oct_journal/jor_12345_comments.Rmd"}
\CommentTok{# start editing, type i to enter insert mode}
\ExtensionTok{i}
\CommentTok{# when editing is done, to exit and save, write and quit}
\CommentTok{# esc first}
\NormalTok{:}\ExtensionTok{wq}
\end{Highlighting}
\end{Shaded}

\hypertarget{vim-cursor-movements}{%
\section{Vim Cursor Movements}\label{vim-cursor-movements}}

Move cursor to the beginning of a line? This should not be done in edit
mode, esc, use movement commands, then back to insert mode. There are
six most basic movements, \emph{wbjk()0\$}:

\begin{Shaded}
\begin{Highlighting}[]
\CommentTok{# forward one word}
\FunctionTok{w}
\CommentTok{# back one word to begin}
\ExtensionTok{b}
\CommentTok{# back to word to end}
\ExtensionTok{e}

\CommentTok{# forward one line, achieved by also pressing downward arrow}
\ExtensionTok{j}
\CommentTok{# backward one line, achieved by pressing upward arrow}
\ExtensionTok{k}
\CommentTok{# left arrow}
\ExtensionTok{h}
\CommentTok{# right arrow}
\ExtensionTok{l}

\CommentTok{# Jump forward one Sentence}
\NormalTok{)}
\CommentTok{# Jump backward one Sentence}
\KeywordTok{(}

\CommentTok{# Forward one page and back one page}
\CommentTok{# ctrl+f, ctrl+b}

\CommentTok{# Begin of line}
\ExtensionTok{0}
\CommentTok{# End of Line}
\NormalTok{$}

\CommentTok{# Begin of File}
\ExtensionTok{gg}
\CommentTok{# End of File}
\ExtensionTok{G}
\end{Highlighting}
\end{Shaded}

In combination, the basic commands can be used in combination

\begin{Shaded}
\begin{Highlighting}[]
\CommentTok{# move to the beginning of the next line}
\ExtensionTok{j0}
\CommentTok{# move to the end of the next line}
\ExtensionTok{j}\NormalTok{$}
\end{Highlighting}
\end{Shaded}

\hypertarget{very-convinent-functions}{%
\section{Very Convinent Functions}\label{very-convinent-functions}}

Some functionalities take much more efforts to complete in many other
editors, but are very easy in vim:

\begin{Shaded}
\begin{Highlighting}[]
\CommentTok{# combine lines: bunch of lines, with carriage break, one block of text, reflow/join}
\CommentTok{# 1. select block}
\ExtensionTok{Vjjjj}
\CommentTok{# 2. reflow/join}
\ExtensionTok{J}
\end{Highlighting}
\end{Shaded}

\hypertarget{vim-go-to-spot-in-text-quickly}{%
\subsection{Vim go to Spot in Text
Quickly}\label{vim-go-to-spot-in-text-quickly}}

\begin{verbatim}
# ESC
# Search for word
/abc
# esc
# next occurance of word
n
# last occurance
N
# first occurance overall
ggn
# last occurance overall
G
# clear search highlighting
:noh
\end{verbatim}

back to previous point

\begin{verbatim}
~.
\end{verbatim}

\hypertarget{combo-operations}{%
\subsection{Combo Operations}\label{combo-operations}}

\begin{enumerate}
\def\labelenumi{\arabic{enumi}.}
\tightlist
\item
  Delete:

  \begin{itemize}
  \tightlist
  \item
    delete current word and go into insert mode: \emph{ciw}
  \item
    Delete current word: \emph{dw}
  \item
    Delete from cursor until end of sentence: \emph{dis}
  \item
    Delete from cursor to end of line: \emph{d\$} or \emph{D}
  \item
    Delete until end of paragraph: \emph{dip}
  \item
    Delete all text in file: \emph{ggdG}
  \end{itemize}
\end{enumerate}

\hypertarget{vim-statistics}{%
\subsection{Vim Statistics}\label{vim-statistics}}

\begin{itemize}
\tightlist
\item
  word count: \emph{g, ctrl+g}
\end{itemize}

\hypertarget{vim-errors}{%
\section{Vim Errors}\label{vim-errors}}

Vim seems frozen (hard for vim to freeze, it is not really frozen). This
is likely due to pressing \emph{ctrl + s}, then need to press \emph{ctrl
+ q}.

vim error: ``E297: Write error in swap file''

\begin{verbatim}
# Trying to save
:w
# E297 can not save, override changes elswhere, and force save what is on screen.
:e!
\end{verbatim}

\hypertarget{vim-set-up-others}{%
\section{Vim Set-Up Others}\label{vim-set-up-others}}

\hypertarget{chromebook}{%
\subsection{Chromebook}\label{chromebook}}

pressing esc key is an issue, too far away. On chromebook, remap the
search key which is in the spot of the caps lock key to esc. and remap
the esc key to search key.

\begin{enumerate}
\def\labelenumi{\arabic{enumi}.}
\tightlist
\item
  search for keyboard under settings
\item
  launcher to escape, and escape to launcher
\end{enumerate}

\end{document}
