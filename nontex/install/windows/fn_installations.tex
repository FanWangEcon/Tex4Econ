\PassOptionsToPackage{unicode=true}{hyperref} % options for packages loaded elsewhere
\PassOptionsToPackage{hyphens}{url}
\PassOptionsToPackage{dvipsnames,svgnames*,x11names*}{xcolor}
%
\documentclass[]{article}
\usepackage{lmodern}
\usepackage{amssymb,amsmath}
\usepackage{ifxetex,ifluatex}
\usepackage{fixltx2e} % provides \textsubscript
\ifnum 0\ifxetex 1\fi\ifluatex 1\fi=0 % if pdftex
  \usepackage[T1]{fontenc}
  \usepackage[utf8]{inputenc}
  \usepackage{textcomp} % provides euro and other symbols
\else % if luatex or xelatex
  \usepackage{unicode-math}
  \defaultfontfeatures{Ligatures=TeX,Scale=MatchLowercase}
\fi
% use upquote if available, for straight quotes in verbatim environments
\IfFileExists{upquote.sty}{\usepackage{upquote}}{}
% use microtype if available
\IfFileExists{microtype.sty}{%
\usepackage[]{microtype}
\UseMicrotypeSet[protrusion]{basicmath} % disable protrusion for tt fonts
}{}
\IfFileExists{parskip.sty}{%
\usepackage{parskip}
}{% else
\setlength{\parindent}{0pt}
\setlength{\parskip}{6pt plus 2pt minus 1pt}
}
\usepackage{xcolor}
\usepackage{hyperref}
\hypersetup{
            colorlinks=true,
            linkcolor=Maroon,
            filecolor=Maroon,
            citecolor=Blue,
            urlcolor=blue,
            breaklinks=true}
\urlstyle{same}  % don't use monospace font for urls
\usepackage[margin=1in]{geometry}
\usepackage{color}
\usepackage{fancyvrb}
\newcommand{\VerbBar}{|}
\newcommand{\VERB}{\Verb[commandchars=\\\{\}]}
\DefineVerbatimEnvironment{Highlighting}{Verbatim}{commandchars=\\\{\}}
% Add ',fontsize=\small' for more characters per line
\usepackage{framed}
\definecolor{shadecolor}{RGB}{248,248,248}
\newenvironment{Shaded}{\begin{snugshade}}{\end{snugshade}}
\newcommand{\AlertTok}[1]{\textcolor[rgb]{0.94,0.16,0.16}{#1}}
\newcommand{\AnnotationTok}[1]{\textcolor[rgb]{0.56,0.35,0.01}{\textbf{\textit{#1}}}}
\newcommand{\AttributeTok}[1]{\textcolor[rgb]{0.77,0.63,0.00}{#1}}
\newcommand{\BaseNTok}[1]{\textcolor[rgb]{0.00,0.00,0.81}{#1}}
\newcommand{\BuiltInTok}[1]{#1}
\newcommand{\CharTok}[1]{\textcolor[rgb]{0.31,0.60,0.02}{#1}}
\newcommand{\CommentTok}[1]{\textcolor[rgb]{0.56,0.35,0.01}{\textit{#1}}}
\newcommand{\CommentVarTok}[1]{\textcolor[rgb]{0.56,0.35,0.01}{\textbf{\textit{#1}}}}
\newcommand{\ConstantTok}[1]{\textcolor[rgb]{0.00,0.00,0.00}{#1}}
\newcommand{\ControlFlowTok}[1]{\textcolor[rgb]{0.13,0.29,0.53}{\textbf{#1}}}
\newcommand{\DataTypeTok}[1]{\textcolor[rgb]{0.13,0.29,0.53}{#1}}
\newcommand{\DecValTok}[1]{\textcolor[rgb]{0.00,0.00,0.81}{#1}}
\newcommand{\DocumentationTok}[1]{\textcolor[rgb]{0.56,0.35,0.01}{\textbf{\textit{#1}}}}
\newcommand{\ErrorTok}[1]{\textcolor[rgb]{0.64,0.00,0.00}{\textbf{#1}}}
\newcommand{\ExtensionTok}[1]{#1}
\newcommand{\FloatTok}[1]{\textcolor[rgb]{0.00,0.00,0.81}{#1}}
\newcommand{\FunctionTok}[1]{\textcolor[rgb]{0.00,0.00,0.00}{#1}}
\newcommand{\ImportTok}[1]{#1}
\newcommand{\InformationTok}[1]{\textcolor[rgb]{0.56,0.35,0.01}{\textbf{\textit{#1}}}}
\newcommand{\KeywordTok}[1]{\textcolor[rgb]{0.13,0.29,0.53}{\textbf{#1}}}
\newcommand{\NormalTok}[1]{#1}
\newcommand{\OperatorTok}[1]{\textcolor[rgb]{0.81,0.36,0.00}{\textbf{#1}}}
\newcommand{\OtherTok}[1]{\textcolor[rgb]{0.56,0.35,0.01}{#1}}
\newcommand{\PreprocessorTok}[1]{\textcolor[rgb]{0.56,0.35,0.01}{\textit{#1}}}
\newcommand{\RegionMarkerTok}[1]{#1}
\newcommand{\SpecialCharTok}[1]{\textcolor[rgb]{0.00,0.00,0.00}{#1}}
\newcommand{\SpecialStringTok}[1]{\textcolor[rgb]{0.31,0.60,0.02}{#1}}
\newcommand{\StringTok}[1]{\textcolor[rgb]{0.31,0.60,0.02}{#1}}
\newcommand{\VariableTok}[1]{\textcolor[rgb]{0.00,0.00,0.00}{#1}}
\newcommand{\VerbatimStringTok}[1]{\textcolor[rgb]{0.31,0.60,0.02}{#1}}
\newcommand{\WarningTok}[1]{\textcolor[rgb]{0.56,0.35,0.01}{\textbf{\textit{#1}}}}
\usepackage{graphicx,grffile}
\makeatletter
\def\maxwidth{\ifdim\Gin@nat@width>\linewidth\linewidth\else\Gin@nat@width\fi}
\def\maxheight{\ifdim\Gin@nat@height>\textheight\textheight\else\Gin@nat@height\fi}
\makeatother
% Scale images if necessary, so that they will not overflow the page
% margins by default, and it is still possible to overwrite the defaults
% using explicit options in \includegraphics[width, height, ...]{}
\setkeys{Gin}{width=\maxwidth,height=\maxheight,keepaspectratio}
\setlength{\emergencystretch}{3em}  % prevent overfull lines
\providecommand{\tightlist}{%
  \setlength{\itemsep}{0pt}\setlength{\parskip}{0pt}}
\setcounter{secnumdepth}{5}
% Redefines (sub)paragraphs to behave more like sections
\ifx\paragraph\undefined\else
\let\oldparagraph\paragraph
\renewcommand{\paragraph}[1]{\oldparagraph{#1}\mbox{}}
\fi
\ifx\subparagraph\undefined\else
\let\oldsubparagraph\subparagraph
\renewcommand{\subparagraph}[1]{\oldsubparagraph{#1}\mbox{}}
\fi

% set default figure placement to htbp
\makeatletter
\def\fps@figure{htbp}
\makeatother


\title{New Computer Fan Data-Science Set-up\\
Python (Anaconda), R (Rstudio), Matlab and Latex (Texlive)\\
Atom, VScode, Pycharm}
\author{}
\date{\vspace{-2.5em}}

\begin{document}
\maketitle

Go back to \href{http://fanwangecon.github.io/}{fan}'s
\href{http://fanwangecon.github.io/Tex4Econ/}{Tex4Econ and Miscellaneous
Repository}.

\hypertarget{objective}{%
\section{Objective}\label{objective}}

Install various compilers and IDE:

\begin{enumerate}
\def\labelenumi{\arabic{enumi}.}
\tightlist
\item
  Python (Anaconda, Pycharm + VSCode + Atom)
\item
  R (VSCode + Atom)
\item
  Matlab
\item
  Latex and PDF (texlive, Okular, VSCode + Atom + Texstudio)
\end{enumerate}

Test at the end:

\begin{enumerate}
\def\labelenumi{\arabic{enumi}.}
\tightlist
\item
  Do sample \emph{py} files work?

  \begin{itemize}
  \tightlist
  \item
    work from command line
  \item
    work from atom (hydrogen), Pycharm, VSCode
  \end{itemize}
\item
  Do sample \emph{rmd} files work?

  \begin{itemize}
  \tightlist
  \item
    work from r-studio
  \item
    work from atom (hydrogen), VSCode
  \end{itemize}
\item
  Do tex templates compile?

  \begin{itemize}
  \tightlist
  \item
    tikz files, template files with bib
  \item
    work from texstudio, VSCode live compile, Atom
  \end{itemize}
\end{enumerate}

Installation Plan outline:

\begin{enumerate}
\def\labelenumi{\arabic{enumi}.}
\tightlist
\item
  Conda install (python)
\item
  Install Atom along with Hydrogen and other packages.
\end{enumerate}

\hypertarget{python-and-anaconda-installation}{%
\section{Python and Anaconda
Installation}\label{python-and-anaconda-installation}}

Use conda across platforms, so that locally on windows and ubuntu and
remotely on aws, can have the same software setup environment.

\begin{itemize}
\tightlist
\item
  Search for Anaconda Prompt, right click, choose run as administrator.
\item
  Check software versions.
\end{itemize}

\begin{Shaded}
\begin{Highlighting}[]
\ExtensionTok{conda}\NormalTok{ list anaconda}
\ExtensionTok{python}\NormalTok{ -V}
\end{Highlighting}
\end{Shaded}

\hypertarget{first-time-conda-install}{%
\subsection{First Time Conda Install}\label{first-time-conda-install}}

\href{https://www.anaconda.com/distribution/}{Download Anaconda Python
3} and install for all users. Afterwards, Anaconda does not
automatically get added to Windows Path. Need to use Anaconda Prompt to
access programs. To access Anaconda packages from windows prompt, from
git bash, from R, etc, need to Add Anaconda to Windows Path.

To install conda in Linux/Debian, follow
\href{https://fanwangecon.github.io/Tex4Econ/nontex/install/linux/fn_ubuntu.html}{these
instructions}. Overall, installations are very similar.

In command/anaconda prompt:

\begin{Shaded}
\begin{Highlighting}[]
\CommentTok{# To remove conda Fully}
\FunctionTok{rm}\NormalTok{ -rf ~/anaconda3}

\ExtensionTok{where}\NormalTok{ anaconda}
\CommentTok{# C:/ProgramData/Anaconda3/Scripts/anaconda.exe}
\ExtensionTok{where}\NormalTok{ python}
\CommentTok{# C:/ProgramData/Anaconda3/python.exe}
\ExtensionTok{where}\NormalTok{ jupyter}
\CommentTok{# C:/ProgramData/Anaconda3/Scripts/jupyter.exe}
\ExtensionTok{where}\NormalTok{ jupyter-kernelspec}
\CommentTok{# if R installed already}
\ExtensionTok{where}\NormalTok{ r}

\CommentTok{###############################}
\CommentTok{# Add these to Windows PATH:}
\CommentTok{###############################}
\CommentTok{# C:\textbackslash{}ProgramData\textbackslash{}Anaconda3\textbackslash{}Scripts}
\CommentTok{# C:\textbackslash{}ProgramData\textbackslash{}Anaconda3}
\end{Highlighting}
\end{Shaded}

To Add Anaconda to Path, In Windows 1. Search for: Environment Variables
2. Edit Environment Variables 3. Add new to Path (lower half): -
C:/ProgramData/Anaconda3/Scripts/ - C:/ProgramData/Anaconda3/ 4. Now
open up regular windows command Prompt, Type in: - conda --version -
also Close and Open up Git Bash: conda --version

\hypertarget{conda-update}{%
\subsection{Conda Update}\label{conda-update}}

Open up Anaconda Navigator, it will update navigator automatically. If
there are errors, might have to clean first.

There are different channels/repositories from which packages could be
installed: \emph{conda}, \emph{conda-forge}, for example. Can change
where to look for packages first. Many packages are in multiple
channels. Conda-forge often have mroe recent packages.

\begin{Shaded}
\begin{Highlighting}[]
\CommentTok{# if there are bugs}
\CommentTok{# conda clean --packages}
\CommentTok{# use conda-forge as main channel, more updated packages}
\ExtensionTok{conda}\NormalTok{ config --get channels}
\ExtensionTok{conda}\NormalTok{ config --add channels conda-forge}
\ExtensionTok{conda}\NormalTok{ config --get channels}
\CommentTok{# remove conda-forge from channels (do so for main env install)}
\ExtensionTok{conda}\NormalTok{ config --remove channels conda-forge}
\CommentTok{# normal update}
\ExtensionTok{conda}\NormalTok{ update --all}

\CommentTok{# install additional packages}
\ExtensionTok{conda}\NormalTok{ install -y statsmodels datashape seaborn}
\ExtensionTok{conda}\NormalTok{ install -c conda-forge -y interpolation awscli}
\ExtensionTok{conda}\NormalTok{ install -c anaconda -y boto3}
\end{Highlighting}
\end{Shaded}

\hypertarget{python-command-line-test}{%
\subsection{Python Command Line Test}\label{python-command-line-test}}

Fist test python under command line:

\begin{Shaded}
\begin{Highlighting}[]
\CommentTok{# windows}
\ExtensionTok{python} \StringTok{"C:/Users/fan/PyFan/ProjectSupport/Testing/Numpy/Functions.py"}
\ExtensionTok{python} \StringTok{"C:/Users/fan/PyFan/ProjectSupport/graph/subplot.py"}
\CommentTok{# linux}
\ExtensionTok{python}\NormalTok{ ~/PyFan/ProjectSupport/Testing/Numpy/Functions.py}
\ExtensionTok{python}\NormalTok{ ~/PyFan/ProjectSupport/graph/subplot.py}
\end{Highlighting}
\end{Shaded}

\hypertarget{r-installation}{%
\section{R Installation}\label{r-installation}}

R could be installed from Anaconda or directly on windows outside of the
Anaconda directories. When installing from within Anaconda, could create
R environment that are isolated from the rest of the computer. The R
environment could be updated and deleted without disturbing dependencies
in conda main or the rest of the computer.

The computer could have multiple R installations. Several in different
conda R environment, and also several under windows/linux primary main
user R directories. Type \emph{which R/where R} to see if you open up R
from command line at the moment, which directory's R will be used.
Inside Conda R environment, it will be a different R if there is a
different R version there.

\hypertarget{fully-uninstall-r}{%
\subsection{Fully Uninstall r}\label{fully-uninstall-r}}

If R was installed in R enviornments, just delete the environment.
Otherwise, use system's uninstaller. Before that, from
terminal/command-prompt, type \emph{R}, and \emph{.libPaths()} to find
paths. Do so inside conda main, outside of conda main, inside different
environments, find all paths. After uninstaller finishes, check in the
libPath folders to see if there are still stuff there, delete all,
delete the folders.

\begin{Shaded}
\begin{Highlighting}[]
\KeywordTok{.libPath}\NormalTok{()}
\CommentTok{# C:/Users/fan/Documents/R/win-library/3.6}
\CommentTok{# C:/Program Files/R/R-3.6.1/library}
\end{Highlighting}
\end{Shaded}

For Linux and for unsintalling inside conda:

\begin{Shaded}
\begin{Highlighting}[]
\CommentTok{# Exit Conda}
\ExtensionTok{conda}\NormalTok{ deactivate}
\CommentTok{# where is R installed outside of Conda}
\FunctionTok{which}\NormalTok{ R}
\CommentTok{# /usr/bin/R}
\CommentTok{# To remove all}
\FunctionTok{sudo}\NormalTok{ apt-get remove r-base}
\FunctionTok{sudo}\NormalTok{ apt-get remove r-base-core}

\CommentTok{# Inside Conda base}
\ExtensionTok{conda}\NormalTok{ activate}
\CommentTok{# Conda r_env}
\ExtensionTok{conda}\NormalTok{ activate r_env}
\CommentTok{# Where is it installed?}
\FunctionTok{which}\NormalTok{ R}
\CommentTok{# /home/wangfanbsg75/anaconda3/bin/R}
\ExtensionTok{conda}\NormalTok{ uninstall r-base}
\end{Highlighting}
\end{Shaded}

\hypertarget{install-r-outside-of-conda}{%
\subsection{Install R (outside of
Conda)}\label{install-r-outside-of-conda}}

Need to install R. Then need to install also an editor/IDE for R.
Rstudio is the dominant R IDE. VSCode and Atom also works well,
especially VSCode as editors. Dramatically better experience for writing
scripts outside of Rstudio which feels unwieldy and slow especially
compared to VSCode.

Overall plan:

\begin{enumerate}
\def\labelenumi{\arabic{enumi}.}
\tightlist
\item
  \href{https://cloud.r-project.org/}{download R}

  \begin{itemize}
  \tightlist
  \item
    for debian:
    \href{https://cran.r-project.org/bin/linux/debian/}{Johannes Ranke}.
    For Linux/Debian installation, crucial to update the
    \emph{source.list} to include sources that have more recent versions
    of R. If not, will get very old R versions that is not compatible
    with many packages.
  \item
    add R to path for Windows. In Windows Path, add for example:
    \emph{C:/Program Files/R/R-3.6.2/bin/x64/} and \emph{C:/Rtools/bin}
  \end{itemize}
\item
  \href{https://rstudio.com/products/rstudio/download/}{download
  R-studio}
\item
  Open R-studio and auto-detect R
\item
  Install additional packages (from commandline)

  \begin{itemize}
  \tightlist
  \item
    various R data-science packages have dependencies, and additional
    not-R programs might need to be installed. Install them as needed,
    for example: \emph{libcurl4-openssl-dev} and \emph{libssl-dev} from
    command prompt first inside Debian.
  \end{itemize}
\end{enumerate}

\hypertarget{linux-r-install}{%
\subsubsection{Linux R Install}\label{linux-r-install}}

For linux/Debian, to Install latest R:

\begin{Shaded}
\begin{Highlighting}[]
\CommentTok{# Go to get latesdebian latest r sources.list}
\FunctionTok{cat}\NormalTok{ /etc/apt/sources.list}
\CommentTok{# Install this First (should already be installed)}
\FunctionTok{sudo}\NormalTok{ apt install dirmngr}

\CommentTok{# Debian R is maintained by Johannes Ranke, copied from https://cran.r-project.org/bin/linux/debian/:}
\ExtensionTok{apt-key}\NormalTok{ adv --keyserver keys.gnupg.net --recv-key }\StringTok{'E19F5F87128899B192B1A2C2AD5F960A256A04AF'}
\CommentTok{# Add to source.list, for debian stretch (9)}
\CommentTok{# sudo su added for security issue as super-user}
\FunctionTok{sudo}\NormalTok{ su -c }\StringTok{"sudo echo 'deb http://cloud.r-project.org/bin/linux/debian stretch-cran35/' >> /etc/apt/sources.list"}
\CommentTok{# if added wrong lines, delete 3rd line}
\FunctionTok{sudo}\NormalTok{ sed }\StringTok{'3d'}\NormalTok{ /etc/apt/sources.list}

\CommentTok{# Update and Install R, should say updated from cloud.r}
\FunctionTok{sudo}\NormalTok{ apt-get update}
\FunctionTok{sudo}\NormalTok{ apt-get install r-base r-base-dev}

\CommentTok{# Also install these, otherwise r-packages do not install}
\CommentTok{# libxml2 seems need for tidymodels}
\FunctionTok{sudo}\NormalTok{ apt-get install libcurl4-openssl-dev}
\FunctionTok{sudo}\NormalTok{ apt-get install libssl-dev}
\FunctionTok{sudo}\NormalTok{ apt-get install libxml2-dev}
\end{Highlighting}
\end{Shaded}

\hypertarget{install-packages-for-r}{%
\subsubsection{Install Packages for R}\label{install-packages-for-r}}

From command prompt, enter R by typeing \emph{R}, and install packages:

\begin{Shaded}
\begin{Highlighting}[]
\CommentTok{# Can enter R from Command Prompt Conda ENV, much faster than from r-studio}
\CommentTok{# Install R-tools, RTOOLS is for WINDOWS}
\KeywordTok{install.packages}\NormalTok{(}\StringTok{"installr"}\NormalTok{)}
\KeywordTok{library}\NormalTok{(}\StringTok{"installr"}\NormalTok{)}
\KeywordTok{install.Rtools}\NormalTok{()}
\CommentTok{# c:/Rtools/bin;}
\CommentTok{# c:/Rtools/mingw_32/bin;}
\CommentTok{# c:/Rtools/mingw_64/bin;}

\CommentTok{# main packagevps}
\CommentTok{# tidymodel installation on linux difficult}
\KeywordTok{install.packages}\NormalTok{(}\KeywordTok{c}\NormalTok{(}\StringTok{"tidyverse"}\NormalTok{, }\StringTok{"tidymodels"}\NormalTok{, }\StringTok{"tidyr"}\NormalTok{))}
\CommentTok{# development packages}
\KeywordTok{install.packages}\NormalTok{(}\KeywordTok{c}\NormalTok{(}\StringTok{"devtools"}\NormalTok{, }\StringTok{"IRkernel"}\NormalTok{, }\StringTok{"pkgdown"}\NormalTok{, }\StringTok{"roxygen2"}\NormalTok{, }\StringTok{"kableExtra"}\NormalTok{))}
\CommentTok{# other packages}
\KeywordTok{install.packages}\NormalTok{(}\KeywordTok{c}\NormalTok{(}\StringTok{"AER"}\NormalTok{, }\StringTok{"minpack.lm"}\NormalTok{, }\StringTok{"knitr"}\NormalTok{, }\StringTok{"matlab"}\NormalTok{))}

\CommentTok{# Install my package, need to prepare package to work for ubuntu}
\CommentTok{# if install does not work, load locally}
\NormalTok{devtools}\OperatorTok{::}\KeywordTok{install_github}\NormalTok{(}\StringTok{"fanwangecon/R4Econ"}\NormalTok{)}

\CommentTok{# Kernel}
\KeywordTok{install.packages}\NormalTok{(}\KeywordTok{c}\NormalTok{(}\StringTok{'repr'}\NormalTok{, }\StringTok{'IRdisplay'}\NormalTok{, }\StringTok{'evaluate'}\NormalTok{, }\StringTok{'crayon'}\NormalTok{, }\StringTok{'pbdZMQ'}\NormalTok{, }\StringTok{'devtools'}\NormalTok{, }\StringTok{'uuid'}\NormalTok{, }\StringTok{'digest'}\NormalTok{))}
\NormalTok{IRkernel}\OperatorTok{::}\KeywordTok{installspec}\NormalTok{()}
\end{Highlighting}
\end{Shaded}

\hypertarget{install-r-inside-conda}{%
\subsection{Install R (inside Conda)}\label{install-r-inside-conda}}

Key packages are generally available in conda's default channel
(official distribution) and also the more frequently updated conda-forge
channel. Prioritize conda-forge to get the latest packages. After adding
\emph{conda-forge} to channels, that will be prioritized, so when
creating a new environment, conda will install first from conda-forge if
package exists there. Try the \emph{r\_env} generation line with and
without first adding \emph{conda-forge} to channel, and see that the
packages installed will have different versions, reflecting versions in
the default channel and in the \emph{conda-forge} channels.

\hypertarget{set-up-conda-r-environment}{%
\subsubsection{Set up Conda R
environment}\label{set-up-conda-r-environment}}

Create a Conda Environment called \emph{r\_env}, and update its
channels.

\begin{itemize}
\tightlist
\item
  \textbf{Location}: Installed environments are shown appear in the
  \emph{envs} folder of \emph{C:/ProgramData/Anaconda3/envs/}. Can be
  easily deleted without disrupting the main base installation for
  conda. This is very important. During installation, could easily run
  into problems, and need to re-install. Much better to only re-install
  a folder. Note that when R is installed in \emph{envs}, it will not
  show up under add or remove programs for uninstallation from there,
  has to be uninstalled, deleted directly here.
\item
  \textbf{Use only in Env}: Note that if we installed r inside r\_env,
  if we type \emph{r} under the base environment, we can not enter r. We
  can only enter r within r\_env. With PATH properly set-up, this
  happens under Conda Prompt, Windows Prompt, etc.
\item
  \textbf{Multiple Pythons}: With the above installation for R, one
  benefit is that if R requires a different version of Python than what
  the base environment uses, the r\_env could have a different python
  version. So type \emph{python} in the base environment as well as
  inside the \emph{r\_env}. This also means potentially there could be
  duplicated installations I think. Look at the python versions under
  \emph{conda list} in the base and in the r-environment.
\item
  \textbf{File Size}: Note that this creates a large installation
  folder, \emph{C:/ProgramData/Anaconda3/envs/r\_env/}, without
  additional packages, is 1.3 GB in size
\end{itemize}

\begin{Shaded}
\begin{Highlighting}[]
\CommentTok{# outside of conda, install some depencies}
\FunctionTok{sudo}\NormalTok{ apt-get install libcurl4-openssl-dev}
\FunctionTok{sudo}\NormalTok{ apt-get install libssl-dev}
\FunctionTok{sudo}\NormalTok{ apt-get install libxml2-dev}

\CommentTok{# install key r packages within a r-environment}
\ExtensionTok{conda}\NormalTok{ env remove -n r_env}
\CommentTok{# start empty environment}
\ExtensionTok{conda}\NormalTok{ create -n r_env}
\CommentTok{# activate env and check, nothing yet}
\ExtensionTok{conda}\NormalTok{ activate r_env}
\ExtensionTok{conda}\NormalTok{ list}

\CommentTok{# check channel and add env specific channel}
\ExtensionTok{conda}\NormalTok{ config --get channels}
\CommentTok{# to get latest r, use conda-forge, which has more recent Rs}
\CommentTok{# conda config --env --add channels conda-forge}
\CommentTok{# channels for base}
\ExtensionTok{conda}\NormalTok{ config --get channels}
\CommentTok{# channels for cur env}
\ExtensionTok{conda}\NormalTok{ config --get channels --env}

\CommentTok{# see all installed environments}
\ExtensionTok{conda}\NormalTok{ env list}
\CommentTok{# activate an environment to use it, if in base, r does not exist, isolated in r_env}
\ExtensionTok{conda}\NormalTok{ activate r_env}
\CommentTok{# see packages in environment}
\ExtensionTok{conda}\NormalTok{ list}
\CommentTok{# to quit}
\ExtensionTok{conda}\NormalTok{ deactivate}
\end{Highlighting}
\end{Shaded}

\hypertarget{install-r-and-some-packages-in-conda}{%
\subsubsection{Install R and Some Packages in
Conda}\label{install-r-and-some-packages-in-conda}}

Either inside a R specific environment, or outside of it, now do
additional installations for R and also R packages.

During installation, the R programs might require downgrading other
programs in the Conda environment that Python for example also uses.
That is why potentially it is safer to install in R environment.
However, when installing in R enviornment, certain packages could have
issues if installer does not realize it is in an environment with
special folder structure, so installation could fail.

\begin{Shaded}
\begin{Highlighting}[]
\CommentTok{# install files}
\ExtensionTok{conda}\NormalTok{ install -n r_env -c r r r-essentials r-tidyverse r-tidymodels -y}

\CommentTok{# trouble with igraph installation, install here from conda}
\CommentTok{# conda install -n r_env r-essentials r-base r-tidyr r-tidyverse r-devtools r-irkernel r-pkgdown r-roxygen2}
\ExtensionTok{conda}\NormalTok{ install -n r_env -c r r-igraph}
\ExtensionTok{conda}\NormalTok{ install -n r_env -c conda-forge igraph}
\ExtensionTok{conda}\NormalTok{ install -n r_env -c r r-rstantools r-rstan r-rstanarm -y}
\end{Highlighting}
\end{Shaded}

\hypertarget{install-additional-packages-inside-the-r}{%
\subsection{Install Additional Packages inside the
R}\label{install-additional-packages-inside-the-r}}

These are packages that I use, these should be installed inside the
environment. Or they could be installed later from inside r-studio.
Installing inside r-studio is a lot better. \emph{tidymodel} has
installation issues sometimes.

In \emph{r\_env}, type in \emph{R}, and then install from inside
terminal's R. For conda env, devtools will be installed and will install
packages inside \emph{envs/r\_env/Lib/R}.

\begin{Shaded}
\begin{Highlighting}[]
\CommentTok{# enter into env}
\CommentTok{# conda activate r_env}
\CommentTok{# enter R or type in rstudio}
\CommentTok{# sicne R is only install inside r_env, has to enter rstudio from r_env}
\CommentTok{# R}
\CommentTok{# rstudio}

\CommentTok{# main packagevps}
\CommentTok{# tidymodel installation on linux difficult}
\KeywordTok{install.packages}\NormalTok{(}\KeywordTok{c}\NormalTok{(}\StringTok{"rstantools"}\NormalTok{, }\StringTok{"rstan"}\NormalTok{, }\StringTok{"rstanarm"}\NormalTok{))}
\KeywordTok{install.packages}\NormalTok{(}\KeywordTok{c}\NormalTok{(}\StringTok{"tidyr"}\NormalTok{, }\StringTok{"tidyverse"}\NormalTok{, }\StringTok{"tidymodels"}\NormalTok{))}

\CommentTok{# tidymodels}
\CommentTok{# development packages}
\KeywordTok{install.packages}\NormalTok{(}\KeywordTok{c}\NormalTok{(}\StringTok{"devtools"}\NormalTok{, }\StringTok{"IRkernel"}\NormalTok{, }\StringTok{"pkgdown"}\NormalTok{, }\StringTok{"roxygen2"}\NormalTok{, }\StringTok{"kableExtra"}\NormalTok{))}
\CommentTok{# other packages}
\KeywordTok{install.packages}\NormalTok{(}\KeywordTok{c}\NormalTok{(}\StringTok{"AER"}\NormalTok{, }\StringTok{"minpack.lm"}\NormalTok{, }\StringTok{"knitr"}\NormalTok{, }\StringTok{"matlab"}\NormalTok{))}

\CommentTok{# Install my package, need to prepare package to work for ubuntu}
\CommentTok{# if install does not work, load locally}
\NormalTok{devtools}\OperatorTok{::}\KeywordTok{install_github}\NormalTok{(}\StringTok{"fanwangecon/R4Econ"}\NormalTok{)}

\CommentTok{# Kernel}
\KeywordTok{install.packages}\NormalTok{(}\KeywordTok{c}\NormalTok{(}\StringTok{'repr'}\NormalTok{, }\StringTok{'IRdisplay'}\NormalTok{, }\StringTok{'evaluate'}\NormalTok{, }\StringTok{'crayon'}\NormalTok{, }\StringTok{'pbdZMQ'}\NormalTok{, }\StringTok{'devtools'}\NormalTok{, }\StringTok{'uuid'}\NormalTok{, }\StringTok{'digest'}\NormalTok{))}
\NormalTok{IRkernel}\OperatorTok{::}\KeywordTok{installspec}\NormalTok{()}
\end{Highlighting}
\end{Shaded}

\hypertarget{r-studio-set-up-download-from-rstudio}{%
\subsection{R-Studio Set-up Download from
Rstudio}\label{r-studio-set-up-download-from-rstudio}}

\begin{enumerate}
\def\labelenumi{\arabic{enumi}.}
\tightlist
\item
  Download R-studio, and install, as normal (not from conda, directly
  from rstudio website latest windows version)
\item
  Open up anaconda prompt, enter \emph{r\_env} environment:
  \emph{activate r\_env}
\item
  cd into rstudio exe file folder: cd ``C:/Program Files/RStudio/bin''
\item
  start r-studio from inside \emph{r\_env}: rstudio.exe
\item
  inside r-studio, check: \emph{.libPaths()}

  \begin{itemize}
  \tightlist
  \item
    ``C:/ProgramData/Anaconda3/envs/r\_env/Lib/R/library''
  \end{itemize}
\end{enumerate}

\begin{Shaded}
\begin{Highlighting}[]
\ExtensionTok{activate}\NormalTok{ r_env}
\BuiltInTok{cd} \StringTok{"C:/Program Files/RStudio/bin"}
\ExtensionTok{rstudio.exe}

\BuiltInTok{cd} \StringTok{"C:/ProgramData/Anaconda3/envs/r_env/Lib/R/library"}
\StringTok{"C:/Program Files/RStudio/bin/rstudio.exe"}
\end{Highlighting}
\end{Shaded}

For linux, note that rstudio offers Ubuntu as well as Debian versions
that can be installed with wget:

\begin{Shaded}
\begin{Highlighting}[]
\OperatorTok{<}\NormalTok{!}\ExtensionTok{--}\NormalTok{ for ubuntu --}\OperatorTok{>}
\FunctionTok{wget} \StringTok{"https://download1.rstudio.org/desktop/bionic/amd64/rstudio-1.2.5033-amd64.deb"}
\OperatorTok{<}\NormalTok{!}\ExtensionTok{--}\NormalTok{ for debian --}\OperatorTok{>}
\FunctionTok{sudo}\NormalTok{ apt install ./rstudio-1.2.5033-amd64.deb}
\FunctionTok{wget} \StringTok{"https://download1.rstudio.org/desktop/debian9/x86_64/rstudio-1.2.5033-amd64.deb"}
\FunctionTok{sudo}\NormalTok{ apt install ./rstudio-1.2.5033-amd64.deb}
\ExtensionTok{conda}\NormalTok{ install -c r rstudio}
\BuiltInTok{cd} \StringTok{"C:/Program Files/RStudio/bin"}
\ExtensionTok{rstudio.exe}

\BuiltInTok{cd} \StringTok{"C:/ProgramData/Anaconda3/envs/r_env/Lib/R/library"}
\StringTok{"C:/Program Files/RStudio/bin/rstudio.exe"}
\end{Highlighting}
\end{Shaded}

Is RSTUDIO pointing to the R installation you want? Note you can open
Rstudio from differen conda env command line, that will use different R
installations. Or also set:

\begin{itemize}
\tightlist
\item
  \emph{export RSTUDIO\_WHICH\_R=/usr/bin/R}
\item
  \emph{export RSTUDIO\_WHICH\_R=/home/wangfanbsg75/anaconda3/bin/R}
\end{itemize}

\hypertarget{r-tests}{%
\subsection{R Tests}\label{r-tests}}

Test the following file to see if we can execute a R file. Do it inside
\emph{r\_env} and inside a \emph{r} session.

\begin{Shaded}
\begin{Highlighting}[]
\CommentTok{# A simple file with summary statistics using tidyverse}
\KeywordTok{source}\NormalTok{(}\StringTok{'C:/Users/fan/R4Econ/summarize/dist/fst_hist_onevar.R'}\NormalTok{)}
\CommentTok{# Another simple file with summary statistics using tidyverse}
\KeywordTok{source}\NormalTok{(}\StringTok{'C:/Users/fan/R4Econ/support/tibble/fs_tib_basics.R'}\NormalTok{)}
\CommentTok{# A file involving estimation}
\KeywordTok{source}\NormalTok{(}\StringTok{'C:/Users/fan/R4Econ/optimization/cesloglin/fst_ces_plan_linlog.R'}\NormalTok{)}

\CommentTok{# C:/Users/fan/R4Econ/summarize/dist/fst_hist_onevar.Rmd}
\CommentTok{# C:/Users/fan/R4Econ/support/tibble/fs_tib_basics.Rmd}
\CommentTok{# C:/Users/fan/R4Econ/optimization/cesloglin/fst_ces_plan_linlog.Rmd}
\end{Highlighting}
\end{Shaded}

\hypertarget{install-latex-and-pdf}{%
\section{Install Latex and PDF}\label{install-latex-and-pdf}}

Use a combination of overleaf and local compile. The goal is to have the
same installation set-up for windows as well as linux. And hopefully,
locally compilable file also can be compiled on overleaf (only use
packages also available on overleaf).

There are several pieces of things to install:

\begin{enumerate}
\def\labelenumi{\arabic{enumi}.}
\tightlist
\item
  latex compiler: pdflatex, etc
\item
  tex distributions: MikTex, texlive

  \begin{itemize}
  \tightlist
  \item
    these folders, once various packages are installed, could be very
    large, many GB
  \end{itemize}
\item
  latex packages from distributions
\item
  editor (gui): Atom, Sublime text, texStudio, etc
\end{enumerate}

\hypertarget{uninstall}{%
\subsection{Uninstall}\label{uninstall}}

\begin{itemize}
\tightlist
\item
  uninstall \emph{MiKTeX}: delete from programs, then delete remaining
  folder in Program Files.
\item
  uninstall \emph{texlive}: go to \emph{c:/texlive}, delete the folder
\end{itemize}

\hypertarget{install}{%
\subsection{Install}\label{install}}

\emph{Install on Windows}

\begin{itemize}
\tightlist
\item
  \href{https://www.texstudio.org/}{Install Texstudio}
\item
  \href{https://www.tug.org/texlive/acquire-netinstall.html}{Install
  texlive}
\end{itemize}

\emph{Install on Ubuntu}

install texlive and texstudio ubuntu/debian.

\begin{Shaded}
\begin{Highlighting}[]
\FunctionTok{sudo}\NormalTok{ apt-get update}
\FunctionTok{sudo}\NormalTok{ apt-get install texlive-full}
\FunctionTok{sudo}\NormalTok{ apt-get install texstudio}
\end{Highlighting}
\end{Shaded}

Test file compilations in TexStudio.

\hypertarget{okular-pdf}{%
\subsection{Okular PDF}\label{okular-pdf}}

Linux: \emph{sudo apt-get install okular} Windows: install okular from
windows app store

\hypertarget{okular-shortcuts-and-features}{%
\subsubsection{Okular Shortcuts and
Features}\label{okular-shortcuts-and-features}}

\textbf{Features}:

\begin{itemize}
\tightlist
\item
  Okular 3 pages per row view:

  \begin{itemize}
  \tightlist
  \item
    view, overview
  \end{itemize}
\item
  Automatic Update when PDF updated elsewhere

  \begin{itemize}
  \tightlist
  \item
    this is one of the key reasons to use okular, when I knit a Rmd file
    to PDF, the currently open Okular PDF file automatically updates and
    does not give file is open error like Acrobat.
  \end{itemize}
\item
  Background color:

  \begin{itemize}
  \tightlist
  \item
    can easily change background color to non-white.
  \end{itemize}
\end{itemize}

\textbf{Shortcuts}:

\begin{itemize}
\tightlist
\item
  Okular configure shortcuts: setting, configure shortcuts
\item
  Okular if menu bar lost: \emph{Ctrl + m}
\item
  Okular comment pane: \emph{F6}
\item
  Okular close left pane: \emph{F7}
\item
  Okular full scrren: \emph{Ctrl + Shift + F}

  \begin{itemize}
  \tightlist
  \item
    unlike acrobat, pdf allows for multiple scrrens
  \end{itemize}
\end{itemize}

\hypertarget{adobe-pdf}{%
\subsection{Adobe PDF}\label{adobe-pdf}}

PDF Adobe Acrobat Installation + go to adobe website, log in using
\href{mailto:fwang26@uh.edu}{\nolinkurl{fwang26@uh.edu}} + go to my
account, choose view and download my apps, choose acrobat \& PDF,
download Acrobat DC + UH account can only be activated on two accounts
at once, so need to kick other computers out temporarily potentially

\hypertarget{install-various-editors}{%
\section{Install Various Editors}\label{install-various-editors}}

Atom, VSCode, Notepad++, Sublime, etc. These are essential editors that
allows for pleasant programming and writing experiences. See here for
\href{http://fanwangecon.github.io/}{fan} instructions on
\href{https://fanwangecon.github.io/Tex4Econ/nontex/install/linux/fn_vim.html}{vim}
installations.

\hypertarget{vscode-installation}{%
\subsection{VSCode Installation}\label{vscode-installation}}

VSCode is much more light-weight than Atom, much faster. On Windows,
download \href{https://code.visualstudio.com/download}{here}.

For Linux:

\begin{Shaded}
\begin{Highlighting}[]
\CommentTok{# Debian Install}
\ExtensionTok{curl}\NormalTok{ https://packages.microsoft.com/keys/microsoft.asc }\KeywordTok{|} \ExtensionTok{gpg}\NormalTok{ --dearmor }\OperatorTok{>}\NormalTok{ microsoft.gpg}
\FunctionTok{sudo}\NormalTok{ install -o root -g root -m 644 microsoft.gpg /usr/share/keyrings/microsoft-archive-keyring.gpg}
\FunctionTok{sudo}\NormalTok{ sh -c }\StringTok{'echo "deb [arch=amd64 signed-by=/usr/share/keyrings/microsoft-archive-keyring.gpg] https://packages.microsoft.com/repos/vscode stable main" > /etc/apt/sources.list.d/vscode.list'}
\FunctionTok{sudo}\NormalTok{ apt-get install apt-transport-https}
\FunctionTok{sudo}\NormalTok{ apt-get update}
\FunctionTok{sudo}\NormalTok{ apt-get install code }\CommentTok{# or code-insiders}

\CommentTok{# Change Font (https://wwFw.reddit.com/r/Crostini/comments/cesxr0/vscode_ui_and_fonts_too_small_heres_how_to_fix_it/)}
\CommentTok{# modify: /usr/share/applications/code.desktop}
\ExtensionTok{nvim}\NormalTok{ /usr/share/applications/code.desktop}
\CommentTok{#replace this line: Exec=/usr/share/code/code --unity-launch %F}
\VariableTok{Exec=}\NormalTok{sommelier }\ExtensionTok{-X}\NormalTok{ --scale=0.8 --dpi=160 /usr/share/code/code }\StringTok{"--unity-launch %F"}
\CommentTok{#replace this line: Exec=/usr/share/code/code --new-window %F}
\VariableTok{Exec=}\NormalTok{sommelier }\ExtensionTok{-X}\NormalTok{ --scale=0.8 --dpi=160 /usr/share/code/code }\StringTok{"--new-window %F"}

\CommentTok{# Open File}
\CommentTok{# low density mode to see cleary}
\ExtensionTok{code}
\end{Highlighting}
\end{Shaded}

\hypertarget{vscode-extensions}{%
\subsubsection{VSCode Extensions}\label{vscode-extensions}}

Press \emph{Ctrl + P}, paste the \emph{ext install} commands below to
install extensions:

\begin{Shaded}
\begin{Highlighting}[]
\CommentTok{# [Microsoft Python Extension](https://marketplace.visualstudio.com/items?itemName=ms-python.python)}
\ExtensionTok{ext}\NormalTok{ install ms-python.python}
\CommentTok{# [Latex Workshop](https://marketplace.visualstudio.com/items?itemName=James-Yu.latex-workshop)}
\ExtensionTok{ext}\NormalTok{ install latex-workshop}
\CommentTok{# [R](https://marketplace.visualstudio.com/items?itemName=Ikuyadeu.r)}
\ExtensionTok{ext}\NormalTok{ install ikuyadeu.r}
\end{Highlighting}
\end{Shaded}

\hypertarget{vscode-shortcuts-and-features}{%
\subsubsection{VSCode Shortcuts and
Features}\label{vscode-shortcuts-and-features}}

\textbf{Features}:

\begin{enumerate}
\def\labelenumi{\arabic{enumi}.}
\tightlist
\item
  open a file, if have workspace where the file is in, automatically
  expand tree and show where file is on left pane.
\end{enumerate}

\textbf{Shortcuts} :

\begin{itemize}
\tightlist
\item
  Zen model: \emph{Ctrl + k, z}, this means press control k, then press
  z
\item
  Open up Settings: \emph{Ctrl + ,}
\item
  Preview Markdown in VSCode: \emph{Ctrl + Shift + V}
\item
  Open up left pane: \emph{Ctrl + B}
\item
  Wrap Text: \emph{alt + Z}, switch between wrap or not to save screen
  space for example.
\end{itemize}

\hypertarget{vscode-and-r}{%
\subsubsection{VSCode and R}\label{vscode-and-r}}

VSCode works pretty well it seems with Rmd files. Rmd files are useful
to work through algorithm, examples in preparation for writing usable
functions. Rstudio has been often very slow for me, and I generally feel
like can not stretch around and work in RStudio IDE.

In Atom, Hydrogen works, but really only works with very basic line by
line testing.

Using VSCode with RMD files seems much quicker, and there is room to
breath.

\begin{enumerate}
\def\labelenumi{\arabic{enumi}.}
\tightlist
\item
  Install R following R installation instructions.
\item
  Then install extension
  \href{https://marketplace.visualstudio.com/items?itemName=Ikuyadeu.r}{R}.
  Microsoft stopped supporting rmd in VSCode:
  \href{https://github.com/microsoft/RTVS}{R Tools for Visual Studio}
\item
  properly set path (might have to call vscode from r env)

  \begin{itemize}
  \tightlist
  \item
    \emph{r.rterm.linux}:
    \emph{/home/wangfanbsg75/anaconda3/envs/r\_env/bin/R}
  \item
    \emph{r.rterm.windows}: \emph{C:/Program
    Files/R/R-3.6.2/bin/x64/R.exe}

    \begin{itemize}
    \tightlist
    \item
      note \emph{R.exe} at the end
    \end{itemize}
  \end{itemize}
\item
  Run:

  \begin{itemize}
  \tightlist
  \item
    ctrl + enter to activate R terminal
  \item
    VSCode 2019 allows for Knit Rmd: \emph{Ctrl + Shift + K}
  \item
    also can run code segment: select lines and \emph{ctrl + enter}
  \end{itemize}
\end{enumerate}

\hypertarget{vscode-and-latex}{%
\paragraph{VSCode and Latex}\label{vscode-and-latex}}

Download
\href{https://marketplace.visualstudio.com/items?itemName=James-Yu.latex-workshop}{Latex
Workshop}.

\hypertarget{atom-installation}{%
\subsection{Atom Installation}\label{atom-installation}}

Atom is slow but has clean look and nice git/github integration.

\hypertarget{atom-extensions}{%
\subsubsection{Atom Extensions}\label{atom-extensions}}

Terminal in windows or linux after installation: \emph{apm install
hydrogen}. apm is the atom package manager:

\begin{Shaded}
\begin{Highlighting}[]
\CommentTok{# Once atom is installed, can use apm to install packages}
\FunctionTok{apm}\NormalTok{ install hydrogen}
\FunctionTok{apm}\NormalTok{ install project-manager}
\FunctionTok{apm}\NormalTok{ install sublime-style-column-selection}
\FunctionTok{apm}\NormalTok{ install minimap}

\CommentTok{# vim distraction free editing}
\FunctionTok{apm}\NormalTok{ install vim-mode-plus}
\FunctionTok{apm}\NormalTok{ install zen}

\CommentTok{# Inside Conda open up atom}
\ExtensionTok{atom}
\end{Highlighting}
\end{Shaded}

\hypertarget{atom-shortcuts-and-features}{%
\subsubsection{Atom Shortcuts and
Features}\label{atom-shortcuts-and-features}}

\textbf{Features}:

\begin{Shaded}
\begin{Highlighting}[]
\CommentTok{# open multiple projects from atom at the same time, after repos synced}
\ExtensionTok{atom}\NormalTok{ ~/fanwangecon.github.io ~/Pyfan ~/Teaching ~/Tex4Econ ~/R4Econ ~/M4Econ ~/Py4Econ}
\CommentTok{# open up any py from from within, try 1+1, does it work?}
\CommentTok{# also use script, try ctrl + shift + b, run whole file (not hydrogen)}
\end{Highlighting}
\end{Shaded}

\textbf{Shortcuts} :

\begin{itemize}
\tightlist
\item
  Git Pane: \emph{Ctrl + 9}, \emph{Ctrl + 8}
\end{itemize}

\hypertarget{atom-and-python}{%
\subsubsection{Atom and Python}\label{atom-and-python}}

Having installed Anaconda, now install Atom and Hydrogen to Test with
Python. For both Jupyter as well as Atom, always open from command
prompt, open from \emph{Anaconda Prompt} with admin rights.

Open up \emph{Jupyter Notebook}, does the python kernel work? If does
not, uninstall and re-install Anaconda.

\begin{Shaded}
\begin{Highlighting}[]
\CommentTok{# start jupyter lab, from prompt}
\ExtensionTok{jupyter}\NormalTok{ lab}
\end{Highlighting}
\end{Shaded}

Try \emph{1+1}.

\hypertarget{atom-and-latex}{%
\subsubsection{Atom and Latex}\label{atom-and-latex}}

Install the following two packages in atom:

\begin{enumerate}
\def\labelenumi{\arabic{enumi}.}
\tightlist
\item
  \href{https://github.com/thomasjo/atom-latex}{latex} package on atom.
\item
  \href{https://atom.io/packages/language-latex}{language-latex} package
  on atom
\end{enumerate}

In Atom Latex setting, note that on different platforms, paths to
texlive are different:

\begin{itemize}
\tightlist
\item
  Path:

  \begin{itemize}
  \tightlist
  \item
    Windows: \emph{C:/texlive/2019/bin/win32}
  \item
    Chromebook Linux: \emph{/usr/bin/latex}
  \end{itemize}
\item
  Which PDF viewer to use:

  \begin{itemize}
  \tightlist
  \item
    download Okular and use \href{https://okular.kde.org/}{Okular}
  \end{itemize}
\end{itemize}

\begin{Shaded}
\begin{Highlighting}[]
\CommentTok{# set TeX path}
\CommentTok{# C:/texlive/2019/bin/win32}

\CommentTok{# Latex, after installation}
\CommentTok{# see where texlive appears under root folder directories}
\FunctionTok{sudo}\NormalTok{ find / -name }\StringTok{"texlive"}
\CommentTok{# see latex version, if texlive is used}
\ExtensionTok{latex}\NormalTok{ -v}
\CommentTok{# see main exe directory}
\FunctionTok{which}\NormalTok{ latex}
\CommentTok{# /usr/bin/latex}
\end{Highlighting}
\end{Shaded}

\hypertarget{other-editors}{%
\subsection{Other Editors}\label{other-editors}}

\hypertarget{notepad}{%
\subsubsection{Notepad++}\label{notepad}}

\href{https://notepad-plus-plus.org/downloads/v7.8.2/}{Notepad++}. Many
old projects have xml folder structures for Notepad++.

\begin{itemize}
\tightlist
\item
  Silently update file changes made in other editors: Settings,
  Preferences, MISC., File Status Auto Detection, Update Silently.
\end{itemize}

\hypertarget{sublime}{%
\subsubsection{Sublime}\label{sublime}}

\href{https://www.sublimetext.com/3}{Sublime}

there are some issues with atom. When a file is edited inside Vim, when
changes are saved, a
\href{https://github.com/atom/atom/issues/17186}{temp file is created in
atom}. Sublime would, however, still show the file with the changes.
This makes dual screen editing very difficult. Sublime text is used for
this situation.

\begin{Shaded}
\begin{Highlighting}[]
\CommentTok{# Sublime Linux}
\FunctionTok{wget}\NormalTok{ -qO - https://download.sublimetext.com/sublimehq-pub.gpg }\KeywordTok{|} \FunctionTok{sudo}\NormalTok{ apt-key add -}
\FunctionTok{sudo}\NormalTok{ apt-get install apt-transport-https}
\BuiltInTok{echo} \StringTok{"deb https://download.sublimetext.com/ apt/stable/"} \KeywordTok{|} \FunctionTok{sudo}\NormalTok{ tee /etc/apt/sources.list.d/sublime-text.list}
\FunctionTok{sudo}\NormalTok{ apt-get update}
\FunctionTok{sudo}\NormalTok{ apt-get install sublime-text}
\CommentTok{# launch from terminal}
\ExtensionTok{subl}
\end{Highlighting}
\end{Shaded}

\emph{Sublime Setup}:

\begin{itemize}
\tightlist
\item
  \href{https://weibeld.net/r/rmd_sublime_package.html}{r and rmd setup}
\end{itemize}

\hypertarget{pycharm}{%
\subsubsection{PyCharm}\label{pycharm}}

\href{https://www.}{pycharm}jetbrains.com/pycharm/)

\hypertarget{other-programs-to-install}{%
\section{Other Programs to Install}\label{other-programs-to-install}}

\hypertarget{package-managers}{%
\subsection{Package Managers}\label{package-managers}}

To install packages in windows more easily:

\begin{itemize}
\tightlist
\item
  \href{https://github.com/lukesampson/scoop/wiki/Quick-Start}{scoop}
\end{itemize}

\begin{Shaded}
\begin{Highlighting}[]
\CommentTok{# in powershell}
\CommentTok{# enter cmd}
\CommentTok{# type powershell in cmd to enter powershell from cmd}
\ExtensionTok{powershell}
\CommentTok{# permission}
\ExtensionTok{set-executionpolicy}\NormalTok{ remotesigned -scope currentuser}
\CommentTok{# install}
\ExtensionTok{iwr}\NormalTok{ -useb get.scoop.sh }\KeywordTok{|} \ExtensionTok{iex}
\end{Highlighting}
\end{Shaded}

\hypertarget{utilities}{%
\subsection{Utilities}\label{utilities}}

\begin{itemize}
\tightlist
\item
  7-zip
\item
  evernote
\end{itemize}

\hypertarget{security}{%
\subsection{Security}\label{security}}

\begin{itemize}
\tightlist
\item
  keepass
\item
  \href{https://protonvpn.com/}{Proton VPN}: unlimited usage free
  version available.
\item
  Express VPN
\end{itemize}

\end{document}
