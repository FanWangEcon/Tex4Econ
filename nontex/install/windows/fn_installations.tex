% Options for packages loaded elsewhere
\PassOptionsToPackage{unicode}{hyperref}
\PassOptionsToPackage{hyphens}{url}
\PassOptionsToPackage{dvipsnames,svgnames*,x11names*}{xcolor}
%
\documentclass[
]{article}
\usepackage{lmodern}
\usepackage{amssymb,amsmath}
\usepackage{ifxetex,ifluatex}
\ifnum 0\ifxetex 1\fi\ifluatex 1\fi=0 % if pdftex
  \usepackage[T1]{fontenc}
  \usepackage[utf8]{inputenc}
  \usepackage{textcomp} % provide euro and other symbols
\else % if luatex or xetex
  \usepackage{unicode-math}
  \defaultfontfeatures{Scale=MatchLowercase}
  \defaultfontfeatures[\rmfamily]{Ligatures=TeX,Scale=1}
\fi
% Use upquote if available, for straight quotes in verbatim environments
\IfFileExists{upquote.sty}{\usepackage{upquote}}{}
\IfFileExists{microtype.sty}{% use microtype if available
  \usepackage[]{microtype}
  \UseMicrotypeSet[protrusion]{basicmath} % disable protrusion for tt fonts
}{}
\makeatletter
\@ifundefined{KOMAClassName}{% if non-KOMA class
  \IfFileExists{parskip.sty}{%
    \usepackage{parskip}
  }{% else
    \setlength{\parindent}{0pt}
    \setlength{\parskip}{6pt plus 2pt minus 1pt}}
}{% if KOMA class
  \KOMAoptions{parskip=half}}
\makeatother
\usepackage{xcolor}
\IfFileExists{xurl.sty}{\usepackage{xurl}}{} % add URL line breaks if available
\IfFileExists{bookmark.sty}{\usepackage{bookmark}}{\usepackage{hyperref}}
\hypersetup{
  pdftitle={New Windows Computer Data-Science Set-up: Python (anaconda, pycharm), R (r-studio) and other Programs Installations},
  colorlinks=true,
  linkcolor=Maroon,
  filecolor=Maroon,
  citecolor=Blue,
  urlcolor=blue,
  pdfcreator={LaTeX via pandoc}}
\urlstyle{same} % disable monospaced font for URLs
\usepackage[margin=1in]{geometry}
\usepackage{color}
\usepackage{fancyvrb}
\newcommand{\VerbBar}{|}
\newcommand{\VERB}{\Verb[commandchars=\\\{\}]}
\DefineVerbatimEnvironment{Highlighting}{Verbatim}{commandchars=\\\{\}}
% Add ',fontsize=\small' for more characters per line
\usepackage{framed}
\definecolor{shadecolor}{RGB}{248,248,248}
\newenvironment{Shaded}{\begin{snugshade}}{\end{snugshade}}
\newcommand{\AlertTok}[1]{\textcolor[rgb]{0.94,0.16,0.16}{#1}}
\newcommand{\AnnotationTok}[1]{\textcolor[rgb]{0.56,0.35,0.01}{\textbf{\textit{#1}}}}
\newcommand{\AttributeTok}[1]{\textcolor[rgb]{0.77,0.63,0.00}{#1}}
\newcommand{\BaseNTok}[1]{\textcolor[rgb]{0.00,0.00,0.81}{#1}}
\newcommand{\BuiltInTok}[1]{#1}
\newcommand{\CharTok}[1]{\textcolor[rgb]{0.31,0.60,0.02}{#1}}
\newcommand{\CommentTok}[1]{\textcolor[rgb]{0.56,0.35,0.01}{\textit{#1}}}
\newcommand{\CommentVarTok}[1]{\textcolor[rgb]{0.56,0.35,0.01}{\textbf{\textit{#1}}}}
\newcommand{\ConstantTok}[1]{\textcolor[rgb]{0.00,0.00,0.00}{#1}}
\newcommand{\ControlFlowTok}[1]{\textcolor[rgb]{0.13,0.29,0.53}{\textbf{#1}}}
\newcommand{\DataTypeTok}[1]{\textcolor[rgb]{0.13,0.29,0.53}{#1}}
\newcommand{\DecValTok}[1]{\textcolor[rgb]{0.00,0.00,0.81}{#1}}
\newcommand{\DocumentationTok}[1]{\textcolor[rgb]{0.56,0.35,0.01}{\textbf{\textit{#1}}}}
\newcommand{\ErrorTok}[1]{\textcolor[rgb]{0.64,0.00,0.00}{\textbf{#1}}}
\newcommand{\ExtensionTok}[1]{#1}
\newcommand{\FloatTok}[1]{\textcolor[rgb]{0.00,0.00,0.81}{#1}}
\newcommand{\FunctionTok}[1]{\textcolor[rgb]{0.00,0.00,0.00}{#1}}
\newcommand{\ImportTok}[1]{#1}
\newcommand{\InformationTok}[1]{\textcolor[rgb]{0.56,0.35,0.01}{\textbf{\textit{#1}}}}
\newcommand{\KeywordTok}[1]{\textcolor[rgb]{0.13,0.29,0.53}{\textbf{#1}}}
\newcommand{\NormalTok}[1]{#1}
\newcommand{\OperatorTok}[1]{\textcolor[rgb]{0.81,0.36,0.00}{\textbf{#1}}}
\newcommand{\OtherTok}[1]{\textcolor[rgb]{0.56,0.35,0.01}{#1}}
\newcommand{\PreprocessorTok}[1]{\textcolor[rgb]{0.56,0.35,0.01}{\textit{#1}}}
\newcommand{\RegionMarkerTok}[1]{#1}
\newcommand{\SpecialCharTok}[1]{\textcolor[rgb]{0.00,0.00,0.00}{#1}}
\newcommand{\SpecialStringTok}[1]{\textcolor[rgb]{0.31,0.60,0.02}{#1}}
\newcommand{\StringTok}[1]{\textcolor[rgb]{0.31,0.60,0.02}{#1}}
\newcommand{\VariableTok}[1]{\textcolor[rgb]{0.00,0.00,0.00}{#1}}
\newcommand{\VerbatimStringTok}[1]{\textcolor[rgb]{0.31,0.60,0.02}{#1}}
\newcommand{\WarningTok}[1]{\textcolor[rgb]{0.56,0.35,0.01}{\textbf{\textit{#1}}}}
\usepackage{graphicx,grffile}
\makeatletter
\def\maxwidth{\ifdim\Gin@nat@width>\linewidth\linewidth\else\Gin@nat@width\fi}
\def\maxheight{\ifdim\Gin@nat@height>\textheight\textheight\else\Gin@nat@height\fi}
\makeatother
% Scale images if necessary, so that they will not overflow the page
% margins by default, and it is still possible to overwrite the defaults
% using explicit options in \includegraphics[width, height, ...]{}
\setkeys{Gin}{width=\maxwidth,height=\maxheight,keepaspectratio}
% Set default figure placement to htbp
\makeatletter
\def\fps@figure{htbp}
\makeatother
\setlength{\emergencystretch}{3em} % prevent overfull lines
\providecommand{\tightlist}{%
  \setlength{\itemsep}{0pt}\setlength{\parskip}{0pt}}
\setcounter{secnumdepth}{5}

\title{New Windows Computer Data-Science Set-up: Python (anaconda, pycharm), R
(r-studio) and other Programs Installations}
\author{}
\date{\vspace{-2.5em}}

\begin{document}
\maketitle

Go back to \href{http://fanwangecon.github.io/}{fan}'s
\href{http://fanwangecon.github.io/Tex4Econ/}{Tex4Econ and Miscellaneous
Repository}.

\hypertarget{conda}{%
\section{Conda}\label{conda}}

Use conda across platforms, so that locally on windows and ubuntu and
remotely on aws, can have the same software setup environment.

Search for Anaconda Prompt, right click, choose run as administrator.

Check software versions.

\begin{Shaded}
\begin{Highlighting}[]
\ExtensionTok{conda}\NormalTok{ list anaconda}
\ExtensionTok{python}\NormalTok{ -V}
\end{Highlighting}
\end{Shaded}

\hypertarget{first-time-conda-install}{%
\subsection{First Time Conda Install}\label{first-time-conda-install}}

Download and install for all users. Afterwards, Anaconda does not
automatically get added to Windows Path. Need to use Anaconda Prompt to
access programs. To access Anaconda packages from windows prompt, from
git bash, from R, etc, need to Add Anaconda to Windows Path

See Anaconda Installation Directory, in anaconda prompt:

\begin{Shaded}
\begin{Highlighting}[]
\ExtensionTok{where}\NormalTok{ anaconda}
\CommentTok{# C:/ProgramData/Anaconda3/Scripts/anaconda.exe}
\ExtensionTok{where}\NormalTok{ python}
\CommentTok{# C:/ProgramData/Anaconda3/python.exe}
\ExtensionTok{where}\NormalTok{ jupyter}
\CommentTok{# C:/ProgramData/Anaconda3/Scripts/jupyter.exe}
\ExtensionTok{where}\NormalTok{ jupyter-kernelspec}
\CommentTok{# if R installed already}
\ExtensionTok{where}\NormalTok{ r}

\CommentTok{###############################}
\CommentTok{# Add these to Windows PATH:}
\CommentTok{###############################}
\CommentTok{# C:/ProgramData/Anaconda3/Scripts}
\CommentTok{# C:/ProgramData/Anaconda3}
\end{Highlighting}
\end{Shaded}

To Add Anaconda to Path, In Windows 1. search for: Environment Variables
2. Edit Environment Variables 3. Add new to Path (lower half): -
C:/ProgramData/Anaconda3/Scripts/ - C:/ProgramData/Anaconda3/ 4. Now
open up regular windows command Prompt, Type in: - conda --version -
also Close and Open up Git Bash: conda --version

\hypertarget{conda-update}{%
\subsection{Conda Update}\label{conda-update}}

Open up Anaconda Navigator, it will update navigator automatically. If
there are errors, might have to clean first.

\begin{Shaded}
\begin{Highlighting}[]
\CommentTok{# # if there are bugs}
\CommentTok{# conda clean --packages}
\CommentTok{# use conda-forge as main channel, more updated packages}
\ExtensionTok{conda}\NormalTok{ config --add channels conda-forge}
\CommentTok{# normal update}
\ExtensionTok{conda}\NormalTok{ update --all}
\end{Highlighting}
\end{Shaded}

\hypertarget{conda-test-python}{%
\subsection{Conda Test Python}\label{conda-test-python}}

\begin{Shaded}
\begin{Highlighting}[]
\CommentTok{# windows}
\NormalTok{python }\StringTok{"C:/Users/fan/PyFan/ProjectSupport/Testing/Numpy/Functions.py"}
\NormalTok{python }\StringTok{"C:/Users/fan/PyFan/ProjectSupport/graph/subplot.py"}
\CommentTok{# linux}
\NormalTok{python }\OperatorTok{~}\ErrorTok{/}\NormalTok{PyFan}\OperatorTok{/}\NormalTok{ProjectSupport}\OperatorTok{/}\NormalTok{Testing}\OperatorTok{/}\NormalTok{Numpy}\OperatorTok{/}\NormalTok{Functions.py}
\NormalTok{python }\OperatorTok{~}\ErrorTok{/}\NormalTok{PyFan}\OperatorTok{/}\NormalTok{ProjectSupport}\OperatorTok{/}\NormalTok{graph}\OperatorTok{/}\NormalTok{subplot.py}
\end{Highlighting}
\end{Shaded}

\hypertarget{additional-packages}{%
\subsection{Additional Packages}\label{additional-packages}}

\begin{Shaded}
\begin{Highlighting}[]
\NormalTok{conda install statsmodels datashape seaborn}
\CommentTok{# conda install statsmodels}
\CommentTok{# conda install datashape}
\CommentTok{# conda install seaborn}

\NormalTok{conda install }\OperatorTok{-}\NormalTok{c conda}\OperatorTok{-}\NormalTok{forge interpolation awscli}
\CommentTok{# conda install -c conda-forge interpolation}
\CommentTok{# conda install -c conda-forge awscli}

\NormalTok{conda install }\OperatorTok{-}\NormalTok{c anaconda boto3}

\NormalTok{conda install }\OperatorTok{-}\NormalTok{c r r}\OperatorTok{-}\NormalTok{irkernel}
\end{Highlighting}
\end{Shaded}

\hypertarget{r-installation}{%
\section{R Installation}\label{r-installation}}

Try to install R from within Conda. The procedure here is the same for
windows as well as linux machines.

\hypertarget{fully-uninstall-r}{%
\subsection{Fully uninstall r}\label{fully-uninstall-r}}

open up RStudio, and see where paths are. also open up R from command
prompt, see where paths are.

\begin{Shaded}
\begin{Highlighting}[]
\KeywordTok{.libPath}\NormalTok{()}
\end{Highlighting}
\end{Shaded}

Open up install and uninstall programs, find R and Rstudio, and
uninstall them. Then delete all files found path paths shown by
\emph{.libPath()}, looking at these: -
\emph{C:/Users/fan/Documents/R/win-library/3.6} - \emph{C:/Program
Files/R/R-3.6.1/library}

\hypertarget{install-r-outside-of-conda}{%
\subsection{Install R (outside of
Conda)}\label{install-r-outside-of-conda}}

\begin{enumerate}
\def\labelenumi{\arabic{enumi}.}
\tightlist
\item
  \href{https://cloud.r-project.org/}{download R}
\item
  \href{https://rstudio.com/products/rstudio/download/}{download
  R-studio}
\item
  Open R-studio and auto-detect R
\item
  Install packages
\end{enumerate}

note that installation will not work if the machine's version of R is
not up to date, this could be the case with default R versions for some
linux installations.

\begin{Shaded}
\begin{Highlighting}[]
\CommentTok{# Install R-tools}
\KeywordTok{install.packages}\NormalTok{(}\StringTok{"installr"}\NormalTok{)}
\KeywordTok{library}\NormalTok{(}\StringTok{"installr"}\NormalTok{)}
\KeywordTok{install.Rtools}\NormalTok{()}
\CommentTok{# c:\textbackslash{}Rtools\textbackslash{}bin;}
\CommentTok{# c:\textbackslash{}Rtools\textbackslash{}mingw_32\textbackslash{}bin;}
\CommentTok{# c:\textbackslash{}Rtools\textbackslash{}mingw_64\textbackslash{}bin;}

\CommentTok{# main packagevps}
\KeywordTok{install.packages}\NormalTok{(}\KeywordTok{c}\NormalTok{(}\StringTok{"tidyr"}\NormalTok{, }\StringTok{"tidyverse"}\NormalTok{, }\StringTok{"tidymodels"}\NormalTok{))}
\CommentTok{# development packages}
\KeywordTok{install.packages}\NormalTok{(}\KeywordTok{c}\NormalTok{(}\StringTok{"devtools"}\NormalTok{, }\StringTok{"irkernel"}\NormalTok{, }\StringTok{"pkgdown"}\NormalTok{, }\StringTok{"roxygen2"}\NormalTok{))}
\CommentTok{# other packages}
\KeywordTok{install.packages}\NormalTok{(}\KeywordTok{c}\NormalTok{(}\StringTok{"AER"}\NormalTok{, }\StringTok{"minpack.lm"}\NormalTok{, }\StringTok{"knitr"}\NormalTok{, }\StringTok{"kableExtra"}\NormalTok{, }\StringTok{"matlab"}\NormalTok{))}

\CommentTok{# Install my package}
\NormalTok{devtools}\OperatorTok{::}\KeywordTok{install_github}\NormalTok{(}\StringTok{"fanwangecon/R4Econ"}\NormalTok{)}
\end{Highlighting}
\end{Shaded}

\hypertarget{set-up-r-environment-in-conda}{%
\subsection{Set up R-Environment in
Conda}\label{set-up-r-environment-in-conda}}

Conda environments are language agnostic. pip is a package manager for
Python. venv is an environment manager for Python. conda is both a
package and environment manager and is language agnostic. Inside
Anaconda Prompt:

Key packages are generally available in conda's default channel
(official distribution) and also the more frequently updated conda-forge
channel. Prioritize conda-forge to get the latest packages. after adding
\emph{conda-forge} to channels, that will be prioritized, so when
creating a new environment, conda will install first from conda-forge if
package exists there. Try the \emph{r\_env} generation line with and
without first adding \emph{conda-forge} to channel, and see that the
packages installed will have different versions, reflecting versions in
the default channel and in the \emph{conda-forge} channels.

\begin{Shaded}
\begin{Highlighting}[]
\CommentTok{# Set up Channel}
\ExtensionTok{conda}\NormalTok{ config --add channels conda-forge}
\CommentTok{# install key r packages within a r-environment}
\ExtensionTok{conda}\NormalTok{ env remove -n r_env}
\ExtensionTok{conda}\NormalTok{ create -n r_env r-essentials r-base r-tidyr r-tidyverse r-devtools r-irkernel r-pkgdown r-roxygen2}
\CommentTok{# see all installed environments}
\ExtensionTok{conda}\NormalTok{ env list}
\CommentTok{# activate an environment to use it, if in base, r does not exist, isolated in r_env}
\ExtensionTok{conda}\NormalTok{ activate r_env}
\CommentTok{# see packages in environment}
\ExtensionTok{conda}\NormalTok{ list}
\CommentTok{# to quit}
\ExtensionTok{conda}\NormalTok{ deactivate}
\end{Highlighting}
\end{Shaded}

\begin{itemize}
\tightlist
\item
  \textbf{Location}: Installed environments are shown appear in the
  \emph{envs} folder of \emph{C:/ProgramData/Anaconda3/envs/}. Can be
  easily deleted without disrupting the main base installation for
  conda. This is very important. During installation, could easily run
  into problems, and need to re-install. Much better to only re-install
  a folder. Note that when R is installed in \emph{envs}, it will not
  show up under add or remove programs for uninstallation from there,
  has to be uninstalled, deleted directly here.
\item
  \textbf{Use only in Env}: Note that if we installed r inside r\_env,
  if we type \emph{r} under the base environment, we can not enter r. We
  can only enter r within r\_env. With PATH properly set-up, this
  happens under Conda Prompt, Windows Prompt, etc.
\item
  \textbf{Multiple Pythons}: With the above installation for R, one
  benefit is that if R requires a different version of Python than what
  the base environment uses, the r\_env could have a different python
  version. So type \emph{python} in the base environment as well as
  inside the \emph{r\_env}. This also means potentially there could be
  duplicated installations I think. Look at the python versions under
  \emph{conda list} in the base and in the r-environment.
\item
  \textbf{File Size}: Note that this creates a large installation
  folder, \emph{C:/ProgramData/Anaconda3/envs/r\_env/}, without
  additional packages, is 1.3 GB in size
\end{itemize}

\hypertarget{install-additional-packages-inside-the-r-environment}{%
\subsection{Install Additional Packages inside the
R-environment}\label{install-additional-packages-inside-the-r-environment}}

These are packages that I use, these should be installed inside the
environment. Or they could be installed later from inside r-studio.
Installing inside r-studio is a lot better.

This did not work in windows, \emph{tidymodels} failed to install.
Perhaps because of directory issues. \emph{tidymodels} required
\emph{rstanarm}, and \emph{rstanarm} failed to install.

\begin{Shaded}
\begin{Highlighting}[]
\CommentTok{### One line}
\ExtensionTok{conda}\NormalTok{ install -c conda-forge r-aer r-minpack.lm r-knitr r-kableExtra r-matlab}
\CommentTok{# conda install -c conda-forge r-rstanarm}

\CommentTok{#######################################}
\CommentTok{### Estimation}
\CommentTok{#######################################}
\CommentTok{# Estimation Tools}
\CommentTok{# tidymodels does not work fron conda install}
\CommentTok{# Linear Regression Package}
\CommentTok{# https://anaconda.org/conda-forge/r-aer}
\CommentTok{# Nonlinear Regression Package}
\CommentTok{# https://anaconda.org/conda-forge/r-minpack.lm}
\ExtensionTok{conda}\NormalTok{ install -c conda-forge r-aer r-minpack.lm}
\CommentTok{# enter r and use devtools}
\CommentTok{# install.packages("tidymodels")}

\CommentTok{#######################################}
\CommentTok{### Visualiation}
\CommentTok{#######################################}
\ExtensionTok{conda}\NormalTok{ install -c conda-forge r-knitr r-kableExtra}

\CommentTok{#######################################}
\CommentTok{### Utilities}
\CommentTok{#######################################}
\CommentTok{# Allowing for Invoking Matlab Files from Inside R}
\ExtensionTok{conda}\NormalTok{ install -c conda-forge r-matlab}
\end{Highlighting}
\end{Shaded}

Install additional packages from inside R. The devtools install will
install packages inside \emph{envs/r\_env/Lib/R}.

\begin{Shaded}
\begin{Highlighting}[]
\CommentTok{#######################################}
\CommentTok{### Own Package}
\CommentTok{#######################################}
\NormalTok{devtools}\OperatorTok{::}\KeywordTok{install_github}\NormalTok{(}\StringTok{"fanwangecon/R4Econ"}\NormalTok{)}
\CommentTok{# install_dev("cli")}
\end{Highlighting}
\end{Shaded}

\hypertarget{r-studio-set-up-download-from-rstudio}{%
\subsection{R-Studio Set-up Download from
Rstudio}\label{r-studio-set-up-download-from-rstudio}}

\begin{enumerate}
\def\labelenumi{\arabic{enumi}.}
\tightlist
\item
  Download R-studio, and install, as normal (not from conda, directly
  from rstudio website latest windows version)
\item
  Open up anaconda prompt, enter \emph{r\_env} environment:
  \emph{activate r\_env}
\item
  cd into rstudio exe file folder: cd ``C:/Program Files/RStudio/bin''
\item
  start r-studio from inside \emph{r\_env}: rstudio.exe
\item
  inside r-studio, check: \emph{.libPaths()}

  \begin{itemize}
  \tightlist
  \item
    ``C:/ProgramData/Anaconda3/envs/r\_env/Lib/R/library''
  \end{itemize}
\end{enumerate}

\begin{Shaded}
\begin{Highlighting}[]
\ExtensionTok{activate}\NormalTok{ r_env}
\BuiltInTok{cd} \StringTok{"C:/Program Files/RStudio/bin"}
\ExtensionTok{rstudio.exe}

\BuiltInTok{cd} \StringTok{"C:/ProgramData/Anaconda3/envs/r_env/Lib/R/library"}
\StringTok{"C:/Program Files/RStudio/bin/rstudio.exe"}
\end{Highlighting}
\end{Shaded}

For linux, note that rstudio offers Ubuntu as well as Debian versions
that can be installed with wget:

\begin{Shaded}
\begin{Highlighting}[]
\OperatorTok{<}\NormalTok{!}\ExtensionTok{--}\NormalTok{ for ubuntu --}\OperatorTok{>}
\FunctionTok{wget} \StringTok{"https://download1.rstudio.org/desktop/bionic/amd64/rstudio-1.2.5033-amd64.deb"}
\FunctionTok{sudo}\NormalTok{ apt install ./rstudio-1.2.5033-amd64.deb}
\OperatorTok{<}\NormalTok{!}\ExtensionTok{--}\NormalTok{ for debian --}\OperatorTok{>}
\FunctionTok{wget} \StringTok{"https://download1.rstudio.org/desktop/debian9/x86_64/rstudio-1.2.5033-amd64.deb"}
\FunctionTok{sudo}\NormalTok{ apt install ./rstudio-1.2.5033-amd64.deb}
\ExtensionTok{conda}\NormalTok{ install -c r rstudio}
\BuiltInTok{cd} \StringTok{"C:/Program Files/RStudio/bin"}
\ExtensionTok{rstudio.exe}

\BuiltInTok{cd} \StringTok{"C:/ProgramData/Anaconda3/envs/r_env/Lib/R/library"}
\StringTok{"C:/Program Files/RStudio/bin/rstudio.exe"}
\end{Highlighting}
\end{Shaded}

\hypertarget{install-packages-from-r-studio}{%
\subsection{Install Packages from
R-Studio}\label{install-packages-from-r-studio}}

Install addition files inside Rstudio. Easier to debug potentially.
Install rtools as described
\href{https://thecoatlessprofessor.com/programming/cpp/installing-rtools-for-compiled-code-via-rcpp/}{here}:
\href{https://cran.r-project.org/bin/windows/Rtools/}{rtools download}.
Alternatively:

\begin{Shaded}
\begin{Highlighting}[]
\KeywordTok{install.packages}\NormalTok{(}\StringTok{"installr"}\NormalTok{)}
\KeywordTok{library}\NormalTok{(}\StringTok{"installr"}\NormalTok{)}
\CommentTok{# Note that this will generate several pop-up windows}
\KeywordTok{install.Rtools}\NormalTok{()}
\CommentTok{# choose to only install 64 bit toolchain}
\CommentTok{# this installs g++ which is needed for rstanarm}
\CommentTok{# Add these below to windows path}
\CommentTok{# c:/Rtools/bin;}
\CommentTok{# c:/Rtools/mingw_64/bin;}

\ControlFlowTok{if}\NormalTok{ (}\OperatorTok{!}\KeywordTok{require}\NormalTok{(devtools)) \{}
  \KeywordTok{install.packages}\NormalTok{(}\StringTok{"devtools"}\NormalTok{)}
  \KeywordTok{library}\NormalTok{(devtools)}
\NormalTok{\}}
\KeywordTok{library}\NormalTok{(devtools)}
\KeywordTok{install_github}\NormalTok{(}\StringTok{"stan-dev/rstanarm"}\NormalTok{, }\DataTypeTok{args =} \StringTok{"--preclean"}\NormalTok{)}

\KeywordTok{setwd}\NormalTok{(}\StringTok{'C:/ProgramData/Anaconda3/envs/r_env/Lib/R/library'}\NormalTok{)}

\KeywordTok{install.packages}\NormalTok{(}\StringTok{"stan-dev/rstanarm"}\NormalTok{, }\DataTypeTok{dependencies=}\OtherTok{TRUE}\NormalTok{, }\DataTypeTok{INSTALL_opts =} \KeywordTok{c}\NormalTok{(}\StringTok{'--no-lock'}\NormalTok{))}

\CommentTok{# other packages}
\KeywordTok{install.packages}\NormalTok{(}\KeywordTok{c}\NormalTok{(}\StringTok{"tidymodels"}\NormalTok{, }\StringTok{"AER"}\NormalTok{, }\StringTok{"minpack.lm"}\NormalTok{, }\StringTok{"knitr"}\NormalTok{, }\StringTok{"kableExtra"}\NormalTok{, }\StringTok{"matlab"}\NormalTok{))}

\CommentTok{# Install my package}
\NormalTok{devtools}\OperatorTok{::}\KeywordTok{install_github}\NormalTok{(}\StringTok{"fanwangecon/R4Econ"}\NormalTok{)}
\end{Highlighting}
\end{Shaded}

\hypertarget{test-files}{%
\subsubsection{Test Files}\label{test-files}}

Test the following file to see if we can execute a R file. Do it inside
\emph{r\_env} and inside a \emph{r} session.

\begin{Shaded}
\begin{Highlighting}[]
\CommentTok{# A simple file with summary statistics using tidyverse}
\KeywordTok{source}\NormalTok{(}\StringTok{'C:/Users/fan/R4Econ/summarize/dist/fst_hist_onevar.R'}\NormalTok{)}
\CommentTok{# Another simple file with summary statistics using tidyverse}
\KeywordTok{source}\NormalTok{(}\StringTok{'C:/Users/fan/R4Econ/support/tibble/fs_tib_basics.R'}\NormalTok{)}
\CommentTok{# A file involving estimation}
\KeywordTok{source}\NormalTok{(}\StringTok{'C:/Users/fan/R4Econ/optimization/cesloglin/fst_ces_plan_linlog.R'}\NormalTok{)}

\CommentTok{# C:\textbackslash{}Users\textbackslash{}fan\textbackslash{}R4Econ\textbackslash{}summarize\textbackslash{}dist\textbackslash{}fst_hist_onevar.Rmd}
\CommentTok{# C:\textbackslash{}Users\textbackslash{}fan\textbackslash{}R4Econ\textbackslash{}support\textbackslash{}tibble\textbackslash{}fs_tib_basics.Rmd}
\CommentTok{# C:\textbackslash{}Users\textbackslash{}fan\textbackslash{}R4Econ\textbackslash{}optimization\textbackslash{}cesloglin\textbackslash{}fst_ces_plan_linlog.Rmd}
\end{Highlighting}
\end{Shaded}

\hypertarget{install-latex}{%
\section{Install Latex}\label{install-latex}}

Use a combination of overleaf and local compile. The goal is to have the
same installation set-up for windows as well as linux. And hopefully,
locally compilable file also can be compiled on overleaf (only use
packages also available on overleaf).

There are several pieces of things to install:

\begin{enumerate}
\def\labelenumi{\arabic{enumi}.}
\tightlist
\item
  latex compiler: pdflatex, etc
\item
  tex distributions: MikTex, texlive

  \begin{itemize}
  \tightlist
  \item
    these folders, once various packages are installed, could be very
    large, many GB
  \end{itemize}
\item
  latex packages from distributions
\item
  editor (gui): Atom, texStudio, etc
\end{enumerate}

\hypertarget{uninstall}{%
\subsection{Uninstall}\label{uninstall}}

\begin{itemize}
\tightlist
\item
  uninstall \emph{MiKTeX}: delete from programs, then delete remaining
  folder in Program Files.
\item
  uninstall \emph{texlive}: go to \emph{c:/texlive}, delete the folder
\end{itemize}

\hypertarget{install}{%
\subsection{Install}\label{install}}

\emph{Install on Windows}

\begin{itemize}
\tightlist
\item
  \href{https://www.texstudio.org/}{Install Texstudio}
\item
  \href{https://www.tug.org/texlive/acquire-netinstall.html}{Install
  texlive}
\end{itemize}

\emph{Install on Ubuntu}

install texlive and texstudio ubuntu/debian.

\begin{Shaded}
\begin{Highlighting}[]
\FunctionTok{sudo}\NormalTok{ apt-get update}
\FunctionTok{sudo}\NormalTok{ apt-get install texlive-full}
\FunctionTok{sudo}\NormalTok{ apt-get install texstudio}
\end{Highlighting}
\end{Shaded}

\hypertarget{compile-in-atom}{%
\subsection{Compile in Atom}\label{compile-in-atom}}

Install the \href{https://github.com/thomasjo/atom-latex}{latex} package
on atom.

\begin{Shaded}
\begin{Highlighting}[]
\CommentTok{# set TeX path}
\CommentTok{# C:\textbackslash{}texlive\textbackslash{}2019\textbackslash{}bin\textbackslash{}win32}
\end{Highlighting}
\end{Shaded}

\hypertarget{other-programs-to-install}{%
\section{Other Programs to Install}\label{other-programs-to-install}}

\hypertarget{development-programs}{%
\subsection{Development Programs}\label{development-programs}}

\begin{itemize}
\tightlist
\item
  \href{https://www.jetbrains.com/pycharm/}{pycharm}
\end{itemize}

\hypertarget{key-programs}{%
\subsection{Key Programs}\label{key-programs}}

\begin{itemize}
\tightlist
\item
  Adobe Acrobat

  \begin{itemize}
  \tightlist
  \item
    go to adobe website, log in using
    \href{mailto:fwang26@uh.edu}{\nolinkurl{fwang26@uh.edu}}
  \item
    go to my account, choose view and download my apps, choose acrobat
    \& PDF, download Acrobat DC
  \item
    UH account can only be activated on two accounts at once, so need to
    kick other computers out temporarily potentially
  \end{itemize}
\end{itemize}

\hypertarget{utilities}{%
\subsection{Utilities}\label{utilities}}

\begin{itemize}
\tightlist
\item
  7-zip
\end{itemize}

\end{document}
